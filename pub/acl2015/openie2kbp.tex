\Section{mapping}{Adapting OpenIE to a Known Relation Schema}
\begin{table*}
\begin{tabular}{l:lc|l:lc}
  \textbf{KBP Relation} & \textbf{Open IE Relation} & \textbf{PMI$^2$} & \textbf{KBP Relation} & \textbf{Open IE Relation} & \textbf{PMI$^2$}\\
  \hline
  \small{\rel{Org:Founded}}     & \ww{found in} & 1.17             & \small{\rel{Per:Date\_Of\_Birth}}    & \ww{be bear on} & 1.83       \\
                                & \ww{be found in} & 1.15          &                                      & \ww{bear on} & 1.28          \\
  \small{\rel{Org:Dissolved}}   & \ww{buy Chrysler in} & 0.95      & \small{\rel{Per:Date\_Of\_Death}}    & \ww{die on} & 0.70  \\
                                & \ww{membership in}   & 0.60      &                                      & \ww{be assassinate on} & 0.65  \\
  \small{\rel{Org:LOC\_Of\_HQ}} & \ww{in} & 2.12                   & \small{\rel{Per:LOC\_Of\_Birth}}     & \ww{be bear in} & 1.21        \\
                                & \ww{base in} & 1.82              & \small{\rel{Per:LOC\_Of\_Death}}     & \ww{elect president of} & 2.89                \\
  \small{\rel{Org:Member\_Of}}  & \ww{tough away game in} & 1.80   & \small{\rel{Per:Religion}}           & \ww{speak about} & 0.67   \\
                                & \ww{away game in} & 1.80         &                                      & \ww{popular for} & 0.60   \\
  \small{\rel{Org:Parents}}     & \ww{'s bank}     & 1.65          & \small{\rel{Per:Parents}}            & \ww{daughter of}     & 0.54          \\
                                & \ww{also add to} & 1.52          &                                      & \ww{son of} & 1.52          \\
  \small{\rel{Org:Founded\_By}} & \ww{invest fund of} & 1.48       & \small{\rel{Per:LOC\_Residence}} & \ww{of} & 1.48       \\
                                & \ww{own stake besides} & 1.18    &                                      & \ww{independent from} & 1.18    \\
\end{tabular}
\caption{\label{tab:samplemapping}
  A selection of mapped relations between the open IE relation set and the
    target KBP relations. 
  The top one or two relations are shown for each of 7 person and 6 organization
    relations.
  In addition to intuitive mappings (e.g.,\ww{found in} $\rightarrow$ 
    \rel{Org:Founded}), we can note some rare, but high precision pairs
    (e.g., \ww{invest fund of} $\rightarrow$ \rel{Org:Founded\_By}).
  We can also see the noise in distant supervision occasionally permeate 
    the mapping, e.g., with \ww{elect president of} $\rightarrow$ \rel{Per:LOC\_Of\_Death}.
}
\end{table*}


% Intro + kbp schema
A common use case for open IE systems is to adapt them to a
  known relation schema.
As a motivating use-case, we take the KBP relation schema, taken from
  the TAC-KBP challenges.
This consists of 42 relations (41 after the 2012 evaluation) roughly
  aligning to Wikipedia Infoboxes for people and organizations.
  Examples include \rel{Per:Title}, a person's title in an
  organization (President, CEO, etc.); \rel{Per:City\_Of\_Birth}, the
  city a person was born in; or \rel{Org:Founded}, the date an organization
  was founded.

% Why OpenIE?
The natural approach for relation extraction when the relation
  schema is known \textit{a priori} is to train either 
  a supervised\needcite or distantly supervised\needcite extractor
  directly on the relation set.
Nonetheless, there are a number of motivations for using open domain
  relation triples.
% Domain adaptation
The most salient of these is that an open IE system can be adapted
  easily to new relation schemata.
In part, this is useful if the new schema does not have any training data
  to train a supervised (or distantly supervised) classifier on; a relatively
  small and intuitive annotation effort can translate between open IE relations
  and many relations in many schemata.
But, even when training data is available, learning a mapping from open IE
  relations to the given schema is often faster and higher precision than
  training a supervised classifier.

% Treat KBP relations as random variables
We describe our method for mapping from open IE relations to the KBP relation
  making use of our distantly supervised corpus (see \refsec{distsup}).
At a high level, our mapping will be pairs which have a high PMI$^2$ value
  between the open IE and KBP relation.
We treat the existence of a relation between two mentions in a sentence as
  a random variable, conditioned on the NER type signature of the mentions.
For instance, we can describe the probability of the relation 
  \rel{Per:Country\_Of\_Birth} occurring between \ww{Barack Obama} and
  \ww{America} in the sentence
\begin{center}
  \ww{Barack Obama was born in America.}
\end{center}
as:
\begin{equation*}
  p(\rel{Per:Country\_Of\_Birth} \mid \typ{Person}, \typ{Country})
\end{equation*}

% Treat open IE relations as random variables
Similarly, for open IE relations, we can describe the probability of
  \rel{be born in} occurring between \ww{Barack Obama} and \ww{America} as:
\begin{equation*}
  p(\rel{be born in} \mid \typ{Person}, \typ{Country})
\end{equation*}

% Note on RegexNER
Note that the NER tag set used is enriched somewhat to reflect the types used
  in the KBP relations more closely. 
10 additional named entity tags are added to the standard tag set, covering
  additional location types (city, state, country), religious affiliation,
  and some special cases (e.g., title, URL).
These are generated from a fixed gazette, as described in
  \newcite{key:2014angeli-kbp}.

% Compute probabilities
Given a sufficiently large corpus, we can compute these probabilities
  empirically as the number of times we see a relation ($r$)
  in the corpus normalized by the number of times we see any relation with the 
  same type signature ($t_1, t_2$):
\begin{equation*}
  p(r \mid t_1, t_2) = \frac{
    \textrm{count}(r, t_1, t_2)
  }{
    \sum_{r'}\textrm{count}(r', t_1, t_2)
  }
\end{equation*}

% Misc notes
For the KBP relations, these counts are the counts derived from the distantly
  supervised dataset.
With open IE relations, the counts can be computed directly.
Relations which do not have a well defined type signature (i.e., are not
  between named entities) are discarded.
For brevity, going forward we will drop the conditioning on the type signature
  of the relation and assume it to be implicitly present.

% Joint probability
The joint probability between a KBP relation $r_k$ and open IE relation
  $r_o$, $p(r_k, r_o)$, can then be computed analogously as the fraction of
  times these two relations co-occur out of the total number of times any two
  relations co-occur (conditioned on the type signature of the arguments):
\begin{equation*}
  p(r_k, r_o) = \frac{
    \textrm{count}(r_k, r_0)
  }{
    \sum_{r_k', r_o'}\textrm{count}(r_k', r_o')
  }
\end{equation*}

% PMI^2
We can then compute The PMI$^2$ value between the two relations:
\begin{equation*}
  pmi^2(r_k, r_o) = \log \left( \frac{p(r_k, r_o)^2}{p(r_k) \cdot p(r_o)} \right)
\end{equation*}

Note that in addition to being a measure related to PMI, this captures
  a notion similar to \textit{alignment by agreement} 
  \cite{key:2006liang-alignment};
  the formula can be equivalently written as 
  $\log \left[ p(r_k \mid r_o) p(r_o \mid r_k)\right]$.
It is also functionally the same as the JC WordNet distance measure
  \cite{key:1997jc-similarity}.

% Tweaks made in practice
Some sample extractions are given in \reftab{samplemapping}.
For the purposes of creating this mapping, relations involving
  locations (e.g., city of birth and country of birth) have been collapsed,
  along with their NER types, as these can be fully disambiguated by their
  context in the sentence.
If a relation mapped to (\textit{location}) of birth is found in a sentence
  with a city as its object, then we can unambiguously map this to the relation
  \rel{Per:City\_Of\_Birth}.
  
% Compare to Soderland NFL
Prior work on learning mappings from open IE to a relation schema for NFL
  games \cite{key:2010soderland-nfl} showed only modest improvements, even
  when incorporating active learning.
For their task, they used a metric tuned to precision on the training data:
  the fraction of correct versus incorrect extractions each relation produces.
We propose that either the automated mapping of the sort we propose inherently
  requires a very large corpus to gather accurate statistics, or that the
  choice of PMI as a selection method creates a more robust mapping than tuning
  directly to training precision.

