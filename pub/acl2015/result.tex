\Section{result}{Evaluation}
% Introduce KBP
We evaluate our approach in the context of a real-world end-to-end 
  relation extraction task -- the TAC KBP Slot Filling challenge.
In Slot Filling, we are given a large unlabeled corpus of text, a fixed
  schema of relations (see \refsec{mapping}), and a set of
  query entities.
The task is to find all relation triples in the corpus that have as a subject
  the query entity, and as a relation one of the defined relations.
This can be viewed intuitively as populating Wikipedia Infoboxes from a large
  unstructured corpus of text.
%  as the task of automatically filling in structured information
%  -- like Wikipedia Infoboxes -- from a large unstructured corpus of text.
% Motivate KBP
%We use the KBP task to evaluate open IE in a real-world, end-to-end task.
%  where it can compete with a range of other approaches.

% Comparison
We compare our approach to the University of Washington submission to
  TAC-KBP 2013 \cite{key:2013soderland-kbp}.
Their system used OpenIE v4.0 (a successor to Ollie) 
  run over the KBP corpus and then they generated a mapping from the 
  extracted relations to the fixed schema.
% Differences
Unlike our system, Open IE v4.0 employs a semantic role
  component extracting structured SRL frames, alongside a conventional
  open IE system.
%Our system does not make use of such structured information.
Furthermore, the UW submission allows for extracting relations and entities
  from substrings of an open IE triple argument.
For example, from the triple \textit{(Smith; was appointed; acting director of Acme
  Corporation)}, they extract that Smith is employed by Acme Corporation.
We disallow such extractions, passing the burden of finding correct
  precise extractions to the open IE system itself (see \refsec{extraction}).

% Misc differences
%Both the UW submission and our system use an entity linker
%  and a coreference system.
For entity linking, the UW submission uses of Tom Lin's entity linker
  \cite{key:2012lin-el}; our submission uses the Illinois Wikifier 
  \cite{key:2011ratinov-el} without the relational inference component, for
  efficiency.
For coreference, UW uses the Stanford coreference system
  \cite{key:stanford-coref}; we employ a variant of the simple coref system
  described in \cite{key:2014pink-kbp}.
%  which removes the need for syntactic
%  parsing and was shown to not reduce end-to-end recall by a significant
%  amount.

\begin{table}
\begin{center}
\begin{tabular}{lccc}
\hline
\textbf{System}               & \textbf{P}    & \textbf{R}    & \textbf{F$_1$} \\
\hline
UW Official                   & \textbf{69.8} & 11.4          & 19.6 \\
\hline
Ollie                         & 57.4          & 4.8           & 8.9  \\
$~~$ + Nominal Rels           & 57.6          & 10.4          & 17.5 \\
\hline
Our System                    & ~~~~          & ~~~           & ~~~\\
$~~$ - Nominal Rels           & 63.6          & 8.6           & 15.2 \\
$~~$ + Nominal Rels           & 59.6          & 14.1          & 22.7 \\
$~~$ + Alt. Name              & 56.1          & 18.0          & 27.3 \\
$~~$ + Alt. Name + Website    & 57.0          & \textbf{18.8} & \textbf{28.2} \\
\hline
\end{tabular}
\end{center}
\caption{\label{tab:results}
A summary of our results on the end-to-end KBP Slot Filling task.
UW official is the submission made to the 2013 challenge.
The second row is the accuracy of Ollie embedded in our framework,
  and of Ollie evaluted with nominal relations from our system.
%We also include results for Ollie augmented with the nominal 
%  relations from our system.
Lastly, we report our system, our system with nominal relations removed,
  and our system combined with an alternate names detector and rule-based
  website detector.
%We also run Ollie in the same framework as our open IE system
%We compare against an official submission to the challenge using
%  Open IE v4.0 (UW Official), as well as the performance of our system
%  using Ollie or Exemplar as the backing open IE system.
}
\end{table}

% Report results (KBP)
We report our results in \reftab{results}.\footnote{
  All results are reported with the \texttt{anydoc} flag set to true
    in the evaluation script, meaning that only the truth of the extracted
    knowledge base entry and not the associated provenance is scored.
  In absence of human evaluators, this is in order to not penalize our 
    system unfairly for extracting a new correct provenance.
  }
UW Official refers to the official submission in the 2013 challenge;
  we show a 3.1 F$_1$ improvement (to 22.7 F$_1$) over this submission, 
  evaluated using a comparable approach.
A common technique in KBP systems but not employed by the official UW
  submission in 2013 is to add alternate names 
  based on entity linking and coreference.
Additionally, websites are often extracted using heuristic name-matching
  as they are hard to capture with traditional relation extraction 
  techniques.
If we make use of both of these, our end-to-end accuracy becomes 28.2 F$_1$.
%If we add in alternate name relations extracted from entity linking and
%  coreference -- a common technique not employed by UW Official -- then
%  our approach can achieve an end-to-end F$_1$ of 25.7.

% Report Results (Ablation)
We attempt to isolate variance in scores due to other components in a full
  KBP system.
We ran the Ollie open IE system \cite{key:2012mausam-ollie} in an identical
  framework to ours, and report accuracy in \reftab{results}.
Note that when an argument to an Ollie extraction contains a named entity, 
  we take the argument to be that named entity.
The low performance of this system can be partially attributed to its inability
  to extract nominal relations.
To normalize for this, we report results when the Ollie extractions are
  supplemented with the nominal relations produced by our system
  (Ollie + Nominal Rels in \reftab{results}).
In the other direction, we can remove the nominal relation extractions from
  our system.
Our system outperforms Ollie in both cases.

\todo{pr curves}
%The persistent improvement in accuracy suggests that our gains are not
%  entirely from the addition of these relations.

%We report two versions of our system: one making use of alternate names
%  via coref and entity linking, and another using only the relations from
%  our system.
%The second of these is a more fair comparision to the UW Official score,
%  which does not make use of coreference for alternate names.
%
%% Report results (ablation)
%In addition, we run our system with our relation extraction component replaced
%  with Ollie \cite{key:2012mausam-ollie} run over the same corpus.
%These offer a more isolated comparison between the systems, removing
%  other conflating factors in the evaluation.

% Demo
%An anonymous demo of our system available at
%  \url{http://128.12.224.119/openie/}.
