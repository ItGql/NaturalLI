\begin{abstract}
Relation triples produced by
  open domain information extraction (open IE) systems 
  are useful relation extraction, inference, and other information
  extraction tasks.
Traditionally these are extracted using a large set of patterns;
  however, this approach is brittle on out-of-domain text
  and long-range dependencies, and gives no insight into the
  sub-structure of the arguments.
We replace the large pattern set of these systems
  with a small set of patterns for canonically formatted
  sentences, and a classifier which learns to extract
  coherent self-contained clauses from longer sentences.
Furthermore, we run natural logic inference over these short
  clauses to determine the maximally specific arguments for each
  candidate triple.
We show that our approach yields an over 9000\% F$_1$ gain
  over a state-of-the-art open IE system
  on the TAC-KBP 2013 test set.
\end{abstract}


%\begin{abstract}
%Short relation triples produced by
%  open domain information extraction (open IE) systems 
%  are useful for a number of NLP tasks.
%% "top-down entailed" vs "bottom-up eager", with examples.
%%These systems eagerly match any pattern in the surface
%%  form.
%%  encourages short paths and results in output like
%%  "a is friends"
%Traditionally these are extracted
%  \textit{bottom up}, eagerly matching any valid triple
%  pattern.
%This approach often misses longer-range dependencies, and
%  extracts triples which are meaningless without their context.
%%  from surface patterns;
%%  however, this encourages short dependency paths
%%  and is prone to extracting triples which are meaningless
%%  without their context.
%%Rather than searching directly for surface patterns,
%%We observe that these triples are trivial to extract from
%%  sufficiently short sentences, 
%We propose a
%  \textit{top-down} approach, shortening the full sentence
%  into logically entailed fragments.
%%  leveraging textual inference to
%%  find short entailed sentences from a longer utterance.
%%We propose a ``top-down'' approach to extracting these triples
%%We propose casting open IE as the restricted entailment task of finding
%%  short entailed sentences from a longer utterance.
%This allows us both to capture longer range dependencies,
%  and allows the system to be sensitive to the larger context of the
%  sentence.
%%In our approach, 
%We extract self-contained clauses from the original 
%  sentence, and then run natural logic inference to further 
%  shorten each clause; triples can then be trivially extracted
%  from these shortened clauses.
%%These short fragments can themselves be useful, or simple rules
%%  can be applied to produce more conventional open IE triples.
%We show that our approach yields an over 9000\% F$_1$ gain
%  over a state-of-the-art open IE system
%  on the TAC-KBP 2013 test set.
%%We validate our approach on the 2013 TAC-KBP competition, showing
%%  an over 9000\% absolute F$_1$ gain over a submission using
%%  UW's OpenIE 4 system.
%\end{abstract}
