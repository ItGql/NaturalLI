\Section{related}{Related Work}
% Talk about OpenIE
There has been a large body of work on open information extraction.
One line of work includes
  TextRunner \cite{key:2007yates-textrunner},
  ReVerb \cite{key:2011fader-reverb},
  and Ollie \cite{key:2012mausam-ollie}.
The Never Ending Language Learning project \cite{key:2010carlson-nell}
  has a similar aim, iteratively learning more facts from the internet
  from a small seed set of examples.
Exemplar \cite{key:2013mesquita-exemplar} adapts the open IE framework to
  $n$-ary relationships similar to semantic role labelling, but without the
  associated expensive machinery.

% Uses of Open IE
Open IE triples have been used in a number of applications.
For example, learning entailment graphs for new triples
  \cite{key:2011berant-entailment}, and
  matrix factorization for mapping from open IE to structured schema
  \cite{key:2012yao-schemas,key:2013riedel-schemas}.

% Talk about relation extraction
Prior work on the KBP challenge can be categorized into a number of approaches.
The most common of these are \textit{distantly supervised} relation extractors
  \cite{key:1999craven-distsup,key:2007wu-distsup,key:2009mintz-distsup,key:2011sun-kbp},
  and rule based systems
  \cite{key:1997soderland-kbp,key:2010grishman-kbp,key:2010chen-kbp}.

% Natural Logic
Prior work has used natural logic \cite{key:2008vanbenthem-natlog} 
  for RTE-style textual entailment \cite{key:2009maccartney-natlog}, as
  a formalism well-suited for formal entailment in neural networks
  \cite{key:2013bowman-natlog}, and as a framework for common-sense reasoning
  \cite{key:2013angeli-truth}.
We adopt the precise semantics of \newcite{key:2014icard-natlog}.
Our approach of finding short entailments from a longer utterance is similar
  in spirit to work on textual entailment for information extraction
  \cite{key:2006romano-ie}.

