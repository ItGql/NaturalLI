\Section{search}{Inference as a Finite State Machine}

% -- FSA Diagrams --
\begin{figure*}[t]
\begin{center}
  \begin{tabular}{cc}
    \resizebox{0.40\textwidth}{!}{\completeFSA} &
      \resizebox{0.40\textwidth}{!}{\collapsedFSA} \\
    (a) & (b)
  \end{tabular}
\end{center}
\caption{
  \label{fig:fsa}
  (a) Natural logic inference expressed as a finite state automaton.
  Omitted edges go to the unknown state (\independent), with the exception of
    omitted edges from $\equivalent$, which go to the state of the edge
    type.
  Green states (\equivalent, \forward) denote valid inferences;
    red states (\alternate, \negate) denote invalid inferences;
    blue states (\reverse, \cover) denote inferences of unknown validity.
  (b) The join table collapsed into the three meaningful states over truth
  values.
}
\end{figure*}

%This work augments the proof system from \refsec{maccartney-proof}
%  in two novel ways:
We show that the tabular proof formulation from \refsec{maccartney-proof}
  can be viewed as a finite state machine,
  and present a novel observation that we can losslessly collapse this
  finite state machine into only three intuitive inference states.
These observations allow us to formulate our search problem such that a search path
  corresponds to an input to (i.e., path through) this collapsed state 
  machine.
%  which in turn specifies a logical derivation.

Taking notation from \refsec{maccartney-proof}, we construct a
  finite state machine over states
  $s \in \{\textrm{\small{\forward}}, \textrm{\small{\reverse}}, \dots \}$.
A machine in state $s_i$ corresponds to relation $s_i$
  holding between the initial premise and the derived fact so far.
States therefore correspond to states of \textit{logical validity}.
The start state is \equivalent.
Outgoing transitions correspond to \textit{inference steps}.
Each transition is labeled with a projected relation
  $\rho(r) \in \{\textrm{\small{\forward}}, \textrm{\small{\reverse}}, \dots\}$,
  and spans from a source
  state $s$ to a target $s'$ according to the join table.
That is, the transition $s \xrightarrow{\rho(r)} s'$ exists iff
  $s' = s \bowtie \rho(r)$.
For example, the path in \reffig{teaser} yields the
  transitions 
  $\equivalent \xrightarrow{\negate} \hspace{-0.5em} \negate \hspace{-0.5em}
               \xrightarrow{\reverse} \alternate
               \xrightarrow{\equivalent} \alternate
               \xrightarrow{\reverse} \alternate$.
%Outgoing transition labels
%  correspond to the projected lexical relation $\rho(r)$;
%  transitions correspond to the join table in \reftab{join}.
%This construction follows from our definition of
%  $s_i \coloneqq s_{i-1} \bowtie \rho(r_i)$:
%  for all possible pairs of state $s$ and relation $r$, we construct a
%  transition $s \xrightarrow{r} s'$ according to $s' = s \bowtie r$.
\reffig{fsa}a shows the automaton, with trivial edges
  omitted for clarity.

Our second contribution is collapsing this automaton into the three
  meaningful states we use as output: 
    \textit{valid} ($\varphi \Rightarrow \psi$),
    \textit{invalid} ($\varphi \Rightarrow \lnot \psi$),
  and \textit{unknown validity} ($\varphi \nRightarrow \psi$).
We can cluster states in \reffig{fsa}a into these three categories.
The relations \equivalent\ and \forward\ correspond to valid inferences;
  \negate\ and \alternate\ correspond to invalid inferences;
  \reverse, \cover\ and \independent\ correspond to unknown validity.
This clustering mirrors that used by MacCartney for his textual
  entailment experiments.

Collapsing the FSA into the form in \reffig{fsa}b becomes straightforward
  from observing the regularities in \reffig{fsa}a.
Nodes in the valid cluster transition to invalid nodes
  always and only on the relations \negate\ and \alternate.
Symmetrically, invalid nodes transition to valid nodes always and only
  on \negate\ and \cover.
A similar pattern holds for the other transitions.

Formally, for every relation $r$ and nodes $a_1$ and $a_2$ in
  the same cluster, if we have transitions 
  $a_1 \xrightarrow{r} b_1$ and $a_2 \xrightarrow{r} b_2$
  then $b_1$ and $b_2$ are necessarily in the same cluster.
As a concrete example, we can take $r = \negate$ and
  the two states in the \textit{invalid} cluster:
  $a_1 = \negate$, $a_2 = \alternate$.
Although $\negate \xrightarrow{\negate} \equivalent$ and
  $\alternate \xrightarrow{\negate} \forward$, both
  \equivalent\ and \forward\ are in the same cluster (\textit{valid}).
It is not trivial \textit{a priori} that the join table should have
  this regularity, and it certainly simplifies the logic for
  inference tasks.
%  future work in Natural Logic can use a join table with only
%  three rows, rather than the seven shown.

A few observations deserve passing remark.
First, even though the
  states \reverse\ and \cover\ appear meaningful, in fact there is no
  ``escaping'' these states to either a valid or invalid
  inference.
Second, the hierarchy over relations presented in
  \newcite{key:2012icard-natlog} becomes apparent -- in particular,
  \negate\ always behaves as negation, whereas its two ``weaker''
  versions (\alternate\ and \cover) only behave as negation in certain
  contexts.
Lastly, with probabilistic inference,
  transitioning to the unknown state can be replaced with staying in the
  current state at a (potentially arbitrarily large) cost to the 
  confidence of validity.
This allows us to make use of only two states:
  \textit{valid} and \textit{invalid}.

%The proof system from \refsec{maccartney-proof} can be naturally cast
%  as a finite state automata.
%This is appealing for at least two reasons:
%  first, it is an efficient means of keeping track of only the relevant
%  information when running Natural Logic inference as a search
%  (see \refsec{inference}).
%Second, this formulation makes clear a theoretical contribution of this
%  work: in the case where relevant output of the system is
%  only whether the derivation is \textit{valid}, \textit{invalid},
%  or \textit{unknown},
%  we can collapse the automata losslessly into only these three states.
%This is both computationally convenient, and conceptually elegant,
%  in that it makes many of the opaque patterns in the join table
%  (\reftab{join}) more clear.
%
%In more detail, taking notation from \refsec{maccartney-proof},
%  we can take the states of our FSA to be the
%  $\beta(x_0, x_i)$ values -- the relation between the current fact
%  and the first antecedent.
%The transitions become the projected lexical relations $\beta(x_{i-1}, e_i)$;
%  note that these are deterministic from the mutation and an analysis
%  of the polarity of the lexical item.
%The FSA is then described in \reffig{fsa}a.
%
%We can cluster each of the states in the FSA -- that is, each of the
%  relations between the first and current fact -- into whether this
%  model-theoretic relation corresponds to a valid, invalid, or unknown
%  inference.
%Following \newcite{key:2007maccartney-natlog}, we can cluster
%  \equivalent\ and \forward\ as valid inferences.
%Provably invalid inferences correspond to \alternate\ and \negate.
%All other relations (\reverse, \cover) correspond to inferences of
%  unknown validity.
%
%We note that there is never a case where two states in a cluster have
%  the same outgoing transition go to states in two different clusters.
%That is, for states $x$ and $y$ in the same cluster
%  (i.e., \textit{valid}, \textit{invalid}, \textit{unknown}),
%  if for relation
%  $r$, $x$ transitions to $x'$ and $y$ transitions to $y'$, it is always
%  the case that $x'$ and $y'$ are also in the same cluster.
%
%We can therefore collapse the automata into just three states, as shown
%  in \reffig{fsa}b.
%A few observations deserve passing remark.
%First, it becomes obvious from the collapsed FSA that even though the
%  states \reverse\ and \cover\ appear to track, in fact there is no
%  ``escaping'' these states back into either a valid or invalid
%  inference.
%Second, the hierarchy over relations presented in
%  \newcite{key:2012icard-natlog} becomes apparent -- in particular,
%  \negate\ always behaves as negation, whereas its two ``weaker''
%  versions (\alternate\ and \cover) still behave as negation, but in
%  fewer contexts.
%
%We have not presented both prerequisites for our inference engine
%  formulated as a search problem.
%The lexical relations from \refsec{maccartney} will define the transitions
%  in our search.
%The reformulation of Natural Logic inference as traversing an FSA will
%  allow our search to efficiently track our inference state.
%We proceed to describe the detail of our search problem.
