
\Section{results}{Experiments}
We evaluate our system on two tasks: the FraCaS test suite used
  by \newcite{key:2007maccartney-natlog} and \newcite{key:2008maccartney-natlog},
  as well as for filtering valid and invalid extractions from the
  ReVerb OpenIE system \cite{key:2011fader-reverb}, as per
  \newcite{key:2013angeli-truth}.

%
% FraCaS
%
\Subsection{results-fracas}{FraCaS Entailment Corpus}
The FraCaS textual entailment corpus \cite{key:1996cooper-fracas}
  is a small corpus of entailment problems, aimed at providing a
  comprehensive test of an entailment system's handling of various
  entailment patterns.

Following \newcite{key:2007maccartney-natlog}, we take only a subset
  of the corpus as applicable for Natural Logic.
12 problems were discarded as degenerate -- lacking either an antecedent or
  a consequent.
151 problems were discarded as involving multiple antecedents to
  justify the inference.
Lastly, it should be noted that many of the sections of the corpus
  are not directly applicable to Natural Logic inferences.

\begin{table}
\begin{center}
  \resizebox{0.48\textwidth}{!}{
  \begin{tabular}{|l|l|c|cc|cc|ccc|}
    \hline
    $\mathsection$ & Category & Count & \multicolumn{2}{|c|}{P} & \multicolumn{2}{|c|}{R} & \multicolumn{3}{|c|}{A} \\
                   &          &       & \blue{N} & M08          & \blue{N} & M08          & \blue{N}  & M07 & M08 \\
    \hline
                          % Pme           P07   Rme          R08   Ame         A07  A08
    1 & Quantifiers  & 44 & \blue{91 }  & 95  & \blue{100} & 100 & \blue{95} & 84 & 97 \\
    2 & Plurals      & 24 & \blue{80 }  & 90  & \blue{29 } & 64  & \blue{38} & 42 & 75 \\
    3 & Anaphora     & 6  & \blue{100}  & 100 & \blue{20 } & 60  & \blue{33} & 50 & 50 \\
    4 & Ellipses     & 25 & \blue{100}  & 100 & \blue{5  } & 5   & \blue{28} & 28 & 24 \\
    5 & Adjectives   & 15 & \blue{80 }  & 71  & \blue{66 } & 83  & \blue{73} & 60 & 80 \\
    6 & Comparatives & 16 & \blue{90 }  & 88  & \blue{100} & 89  & \blue{87} & 69 & 81 \\
    7 & Temporal     & 36 & \blue{75 }  & 86  & \blue{53 } & 71  & \blue{52} & 61 & 58 \\
    8 & Verbs        & 8  & \blue{$-$}  & 80  & \blue{0  } & 66  & \blue{25} & 63 & 62 \\
    9 & Attitudes    & 9  & \blue{$-$}  & 100 & \blue{0  } & 83  & \blue{22} & 55 & 89 \\
    \hline
\multicolumn{2}{|l|}{Applicable (1,5,6)}
                     & 75 & \blue{89}   & 89  & \blue{94}  & 94  & \blue{89} & 76 & 90 \\
    \hline
  \end{tabular}
  }
  \caption{
    Results on the FraCaS textual entailment suite.
    N is this work;
    M07 refers to \newcite{key:2007maccartney-natlog};
      M08 refers to \newcite{key:2008maccartney-natlog}.
    The relevant sections of the corpus intended to be handled by this
      system are sections 1, 5, and 6 (not 2 and 9, which are also
      included in M08).
    \label{tab:fracas}
  }
\end{center}
\end{table}

Results on the corpus are given in \reftab{fracas}.
Since the corpus is not a blind test set, the results are presented
  less as a comparison of performance, but rather as a comparison
  of the expressive power of our search-based approach compared with
  MacCartney's align-and-classify approach.
For the experiments, costs were hard-coded to represent a strict
  logical entailment system -- costs corresponding to valid Natural
  Logic mutations were set to a small constant cost; other costs
  were set to infinity.

The results are generally promising.
Note that our system is comparatively crippled in this framework
  along at least two dimensions:
It cannot appeal to the antecedent when constructing the search,
  leading to the introduction of search errors as a new class of
  error.
Second the derivation process itself does not have access to the
  full parse tree of the candidate fact.
Therefore, it is unable to make large edits over constituents,
  and is unable to appeal to the parse in its mutation decisions.
For instance, our system has no notion of prepositional attachment.

Despite this, the system performs comparable to NatLog on the sections
  it intends to capture.
Furthermore, precision is consistently high among all sections, although
  recall is usually significantly lower.
It should also be noted that \newcite{key:2008maccartney-natlog} 
  intend to capture the inferences in sections 2 (plurals)
  and 9 (attitudes), which we omit from our system.

%
% MTurk
%
\Subsection{results-mturk}{ReVerb Filtering}
\Fig{mturk-pr}{0.60}{mturk}{
  A PR curve showing the accuracy of extraction, given the proportion
    of examples we provide an estimate for.
  Imagine the curve keeps going up left of 0.4 -- hopefully that's
    what'll happen by the deadline.
}

To evaluate our system in a more real-world setting, we adopt the
  task of predicting the truth of ReVerb extractions
  as introduced in \newcite{key:2013angeli-truth}.
Equivalently, this can be viewed as filtering out semantically 
  implausible extractions.

For this data, 4000 randomly selected ReVerb extractions were shown to
  Mechanical Turk workers, who ranked each extraction as either
  \textit{correct}, \textit{plausible}, or \textit{implausible}.
Extractions which obtained more \textit{correct} votes than
  \textit{implausible} votes were considered true, and visa versa.
Ties were discarded from the corpus.
The examples are broken up into 1796 training examples ($70\%$ true)
  and 1975 test examples ($65\%$ true).

Results are given in \reffig{mturk}.
They're not very good yet.
Hopefully they'll get better.
