%%%%%%%%%%%%%%%%%%%%%%%%%%%%%%%%%%%%%%%%%%%%%%%%%%%%%%%%%%%%%%%%%%%%%%%%%%%%%%%
% RESULTS
%%%%%%%%%%%%%%%%%%%%%%%%%%%%%%%%%%%%%%%%%%%%%%%%%%%%%%%%%%%%%%%%%%%%%%%%%%%%%%%

%%%%%%%%%%%%%%%%%%% 
% FraCaS INTRO
%%%%%%%%%%%%%%%%%%%
\begin{frame}{Experiments}
\hh{FraCaS Textual Entailment Suite:}
\begin{itemize}
  \item Used in MacCartney and Manning (2007; 2008).
  \item RTE-style problems: is the hypothesis entailed from the premise? \\
    \vspace{0.1cm}
    P: At least three commissioners spend a lot of time at home. \\
    H: \true{At least three commissioners spend time at home.} \\
    \vspace{0.1cm}
    P: At most ten commissioners spend a lot of time at home. \\
    H: \false{At most ten commissioners spend time at home.}
    \vspace{0.1cm}
  \item 9 focused sections; 3 in scope for this work.
\end{itemize}
\vspace{0.5cm}
\pause

\hh{\textbf{Not} a blind test set!}
\begin{itemize}
  \item ``Can we make deep inferences without knowing the premise \textit{a priori}?''
\end{itemize}
\end{frame}

%%%%%%%%%%%%%%%%%%% 
% FraCaS RESULTS
%%%%%%%%%%%%%%%%%%%
\def\a#1{#1}
\def\b#1{\textbf{#1}}
\begin{frame}{FraCaS Results}

\hh{Systems}
\begin{itemize}
\item[] \textbf{M07}: MacCartney and Manning (2007)
\item[] \textbf{M08}: MacCartney and Manning (2008)
  \begin{itemize}
    \item \textit{Classify} entailment after aligning premise and hypothesis.
  \end{itemize}
\item[] \darkblue{\textbf{N}: NaturalLI (this work)}
  \begin{itemize}
    \item \textit{Search} blindly from hypothesis for the premise.
  \end{itemize}
\end{itemize}
\pause

\begin{center}
  \begin{tabular}{llccc}
    \hline
    $\mathsection$ & Category & \multicolumn{3}{c}{Accuracy} \\
                   &          & M07 & M08 & \darkblue{N}  \\
    \hline
                          % Pme           P07   Rme          R08   Ame         A07  A08
    \a{1} & \a{Quantifiers}  & \a{84} & \a{97} & \a{\darkblue{95}}  \\
    \a{5} & \a{Adjectives}   & \a{60} & \a{80} & \a{\darkblue{73}}  \\
    \a{6} & \a{Comparatives} & \a{69} & \a{81} & \a{\darkblue{87}}  \pause \\
    \hline
    \multicolumn{2}{l}{\b{Applicable (1,5,6)}}
                         & \b{76} & \b{90} & \b{\darkblue{89}} \\
    \hline
  \end{tabular}
\end{center}
\end{frame}

%%%%%%%%%%%%%%%%%%% 
% ConceptNet INTRO
%%%%%%%%%%%%%%%%%%%
\begin{frame}{Experiments}
\hh{ConceptNet:}
\begin{itemize}
  \item A semi-curated collection of common-sense facts. \\
    \vspace{0.1cm}
    \true{not all birds can fly} \\
    \true{noses are used to smell} \\
    \true{nobody wants to die} \\
    \true{music is used for pleasure}
    \vspace{0.1cm}
  \item Negatives: ReVerb extractions marked false by Turkers.
  \item Small (1378 train / 1080 test), but fairly broad coverage.
\end{itemize}
\vspace{0.5cm}
\pause

\hh{Our Knowledge Base:}
\begin{itemize}
  \item 270 million lemmatized Ollie extractions.
\end{itemize}
\end{frame}
  
%%%%%%%%%%%%%%%%%%% 
% ConceptNet RESULTS
%%%%%%%%%%%%%%%%%%%
\begin{frame}{ConceptNet Results}
\hh{Systems}
\begin{itemize}
  \item[] \textbf{Direct Lookup}: Lookup by lemmas.
  \item[] \textbf{NaturalLI}: Our system.
  \pause
  \item[] \textbf{NaturalLI Only}: Use only inference (prohibit exact matches).
\end{itemize}
\pause

\begin{center}
  \begin{tabular}{lcc}
    System             & P     & R    \\
    \hline
    Direct Lookup      & 100.0 & \textbf<5-5>{12.1} \\
    \pause
    NaturalLI Only     & 88.8  & 40.1 \\
    NaturalLI          & 90.6  & \textbf<5-5>{49.1} \\
  \end{tabular}
\end{center}
\pause

\begin{itemize}
  \item 4x improvement in recall.
\end{itemize}
\end{frame}
