\Section{related}{Related Work}
% -- OpenIE
% UW
A large body of work is devoted to compiling open-domain knowledge
  bases.
For instance, OpenIE systems
  \cite{key:2007yates-textrunner,key:2011fader-reverb}
  extract concise facts via surface or dependency patterns.
% NELL
In a similar vein, NELL \cite{key:2010carlson-nell,key:2013gardnerpra-nell}
  continuously learns new high-precision facts from the internet.
%% ConceptNet
%The MIT Media Lab's \sys{ConceptNet} project
%  \cite{key:2004liu-conceptnet}
%  has been working on creating a large knowledge base emphasizing
%  common sense facts.

% -- Applications
Many NLP applications query large knowledge bases.
Prominent examples include
  question answering
    \cite{key:2001voorhees-trec},
  semantic parsing
    \cite{key:1996zelle-semantics,key:2007zettlemoyer-semantics,key:2013kwiatkowski-semantics,key:2014berant-semantics},
  and information extraction systems
    \cite{key:2011hoffman-kbp,key:2012surdeanu-mimlre}.
A goal of this work is to improve accuracy on these
  downstream tasks by providing a \textit{probabilistic} knowledge base
  for
%  both explicitly known and 
  likely true facts.

% -- Extending OpenIE
% (people extending OpenIE)
A natural alternative to the approach taken in this paper is to
  extend knowledge bases by inferring and adding new facts directly.
% (misc)
For instance,
  \newcite{key:2006snow-wordnet} present an approach to enriching 
    the WordNet taxonomy;
  \newcite{key:2011tandon-conceptnet} extend ConceptNet with new facts;
  \newcite{key:2010soderland-adapting} use ReVerb extractions to 
    enrich a domain-specific ontology.
% Richard
\newcite{key:2013chen-completion} and \newcite{key:2013socher-completion}
  use Neural Tensor Networks to predict unseen relation triples in
  WordNet and Freebase, following a line of work by
  \newcite{key:2011bordes-completion} and
  \newcite{key:2012jenatton-completion}.
% Universal schemas
\newcite{key:2012yao-schemas} and \newcite{key:2013riedel-schemas}
  present a related line of work, inferring new relations between
  Freebase entities via inference over both Freebase and
  OpenIE relations.
In contrast, this work runs inference over arbitrary text, without 
  restricting itself to a particular set of relations, or even entities.
%This work, however, focuses primarily on common-sense reasoning rather
%  than inferring relations between named entities.


% -- GOFAI
% (intro)
The goal of tackling common-sense reasoning is by no means novel in
  itself.
Work by Reiter and McCarthy \cite{key:1980reiter-logic,key:1980mccarthy-circumscription}
  attempts to reason about the truth of a consequent
  in the absence of strict logical entailment.
Similarly, \newcite{key:1989pearl-probabilistic} presents a framework for
  assigning confidences to inferences which can be reasonably assumed.
Our approach differs from these attempts in part in its use of Natural Logic
  as the underlying inference engine, and more substantially in its
  attempt at creating a broad-coverage system.
More recently, work by \newcite{key:2002schubert-commonsense} and
  \newcite{key:2009durme-commonsense} approach common sense reasoning
  with \textit{episodic logic}; we differ in our focus on inferring
  truth from an arbitrary query, and in making use of longer inferences.
%  dealing with millions of candidate antecedents.

% -- RTE
This work is similar in many ways to work on 
  recognizing textual entailment -- e.g., 
  \newcite{key:2010-schoenmackers-horn}, \newcite{key:2011berant-entailment}.
Work by \newcite{key:2013lewis-entailment} is particularly relevant,
  as they likewise evaluate on the FraCaS suite (Section 1;
  89\% accuracy with gold trees).
They approach entailment by constructing a CCG parse of the query,
  while mapping questions which are paraphrases of each other to the
  same logical form using distributional relation clustering.
However, their system is unlikely to scale to either our large
  database of premises, nor our breadth of relations.

% Paraphrase-based Q/A
\newcite{key:2014fader-openqa} propose a system for question answering
  based on a sequence of paraphrase rewrites followed by a fuzzy query to
  a structured knowledge base.
This work can be thought of as an elegant framework for unifying this
  two-stage process, while explicitly tracking the ``risk'' taken with
  each paraphrase step.
Furthermore, our system is able to explore mutations which are only
  valid in one direction, rather than the bidirectional entailment of
  paraphrases, and does not require a corpus of such paraphrases for
  training.
  
