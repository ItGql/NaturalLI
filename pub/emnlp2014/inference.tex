\Section{inference}{Inference As Search}
In a classical inference setting, a system is provded both the
  query consequent, and a set of antecedents from which to derive
  the validity of the consequent.
Prior work on inference with Natural Logic has assumed this scenario
  \cite{key:2008maccartney-natlog}, and taken the two-step of approach
  of aligning the antecedent and consequent, and classifying each
  aligned segment into a mutation relation.
For common-sense reasoning, however, we are not given a well-defined
  antecedent, but rather have at our disposal a large database of
  true facts from which to derive the query.
This makes the align-and-classify approach of previous work substantially
  less appealing, as the antecedents are not readily available and would
  need to be retrieved from the database.

We therefore approach the problem as a search task: given the
  query consequent, we search over the space of possible facts for
  a valid antecedent in our database.
The nodes in our search problem correspond to candidate facts; the
  edges are candidate mutations; the costs over these edges encode
  the confidence (or likelihood) that the mutation over this edge
  maintains the truth of the antecedent.

We define the problem by specifying the state space,
  the valid transitions, along with the weights of
  the transitions.
We then describe how these weights translate to the confidence of truth
  of a fact, to be used in learning, and a generalization of
  JC similarity \cite{key:1997jc-similarity}.

%
% STATE
%
\Subsection{inference-state}{States}
A state in the graph is defined as a partial path from the consequent to a
  valid antecedent.
That is, a state is primarily parameterized primarily by the surface
  form of the candidate fact at this point in the (reverse) derivation.

This surface form is augmented with both sense and polarity
  information.
Each word is allowed a sense (up to 31 possible senses), or a default
  sense; in addition, 2 bits are reserved for marking polarity
  (monotone, antitone, nonmonotone).
This leaves sufficient space to store each augmented word in a 32 bit
  integer.
Note also that we refer to potentially arbitrary lexical items in
  WordNet as words -- for instance, \textit{fire truck} is treated as
  a single word.

A number of important details deserve attention in our definition of
  a search state:

\paragraph{Lexical mutations}
As per our definition, transitions between states are transitions 
  between facts.
However, the lexical resources for the mutations are over words
  (e.g., \textit{feline} $\rightarrow$ \textit{cat}) rather than
  entire facts.
This motivates the inclusion of an index in our state, denoting the
  index of the word which may be mutated.

Importantly, this also imposes a natural order in the construction of
  potential antecedents during search.
The mutation index is allowed to shift forwards through the
  fact (at no cost), but not backwards.
This helps make search more tractable, at the expense of rare
  esoteric search errors.


\paragraph{Predicting deletions}
Although inserting words in a derivation (deleting words from
  the reversed derivation) is trivial, the other direction is not.
\naive ly, at every state in our search, we must consider every word in
  the vocabulary as a possible antecedent from which the word was deleted
  to produce the current state.

This is largely handled by storing the database of known facts as a
  Trie.
Since we are constructing antecedents left-to-right, as per above,
  we can propose candidate deletable words from the Trie, looking up
  the partial derivation up to the mutation index.
As a special case, when proposing words for the beginning of the
  fact, we take all facts whose second word matches the first word
  of the search state.

\paragraph{Tracking inference state}
The cost of an inference at any given state depends not only on the
  polarity of the lexical item, but also the state of the derivation
  so far.
We therefore augment a fact with whether the derivation is in a
  \textit{valid} or \textit{invalid} state

\paragraph{Polarity tracking}
Mutating quantifiers can change the polarity information on a subset
  of the words in a fact.
Since we do not have the full parse tree at our disposal at search,
  we track a small amount of metadata to guess the scope of the
  mutated quantifier.

%
% TRANSITIONS
%
\Subsection{inference-transitions}{Transitions}
We begin by introducing some terminology.
A \textit{transition template} is a broad class of transitions; for
  instance WordNet hypernymy.
A \textit{transition} or \textit{transition instance} is a particular
  instantiation of a transition template.
For example, the transition from \textit{cat} to \textit{feiline}.
Lastly, an \textit{edge} in the search space connects two facts, which
  are separated by a single transition instance.
For example, an edge exists between 
  \textit{some cats have tails} and \textit{some felines have tails}.

Note that the edges in the search are not constructed a priori --
  just as a theorem prover would not store all possible derivation steps.
It is sufficient to store the transitions, and construct particular
  edges on demand.
At a high level, we include most relations in WordNet as transitions,
  and parameterize insertions and deletions by the part of speech
  of the token being inserted/deleted.
The full table of transitions is given in \reftab{transitions}, along
  with the relation that transition introduces, and an example edge
  in our derivation corresponding to that transition.

\begin{table*}
\begin{center}
  \begin{tabular}{lcl}
    \textbf{Template} & \textbf{Relation} & \textbf{Example edge} \\
    \hline
    WordNet hypernym                     & \forward    & \textit{some cats like milk} \forward\ \textit{some felines like milk} \\
    WordNet hyponym                      & \reverse    & \textit{some cats like milk} \reverse\ \textit{some feilines like milk} \\
    WordNet antonym$^\dagger$            & \alternate  & \textit{all cats like milk} \alternate\ \textit{all cats hate milk} \\
    WordNet synonym/pertainym$^\dagger$  & \equivalent & \textit{some cats like milk} \equivalent\ \textit{some cats enjoy milk} \\
    Distributional nearest neighbor      & \equivalent & \textit{some cats like milk} \equivalent\ \textit{some cats like dairy} \\
    Delete word$^\dagger$                & \forward    & \textit{some tabby cats like milk} \forward\ \textit{some cats like milk} \\
    Add word$^\dagger$                   & \reverse    & \textit{some cats like milk} \reverse\ \textit{some tabby cats like milk} \\
    Quantifier weaken                    & \forward    & \textit{all cats like milk} \forward\ \textit{some cats like milk} \\
    Quantifer strengthen                 & \reverse    & \textit{some cats like milk} \reverse\ \textit{all feilines like milk} \\
    Quantifer negate                     & \negate     & \textit{some cats like milk} \negate\ \textit{no feilines like milk} \\
    Quantifer synonym                    & \equivalent & \textit{some cats like milk} \equivalent\ \textit{a few cats like milk} \\
    Change word sense                    & \equivalent & 
  \end{tabular}
	\caption{
    The edges allowed during inference.
    Entries with a dagger are parameterized by their part-of-speech
      tag, from the restricted list of $\{$noun$,$adjective$,$verb$,$other$\}$.
    The first column describes the type of the transition.
    The set-theoretic relation introduced by each relation is given in
      the second column.
    The third column gives an example of the transition in practice,
      as an edge in the search graph.
		\label{tab:transitions}
	}
\end{center}
\end{table*}

It should be noted that the mapping from transitions to relation
  types is intentionally imprecise.
For instance, clearly nearest neighbors do not preserve equivalence
  (\equivalent); more subtly, while
  \textit{all cats like milk} \alternate\ \textit{all cats hate milk},
  it is not the case that
  \textit{some cats like milk} \alternate\ \textit{some cats hate milk}.\footnote{
    The latter exmaple is a consequence of the projection table in
    \reftab{projectivity} being overly optimistic.
  }
We mitigate this imprecision by introducing a cost for each transition,
  and learning the appropriate value for this cost
  (see \refsec{learning}).

The cost of an edge is parameterized by two values; the cost of an edge
  from fact $(f,v,p)$ with surface form $f$, validity
  $v$ and polarity $p$ to a new fact $(f',v',p')$ using a transition
  instance $t_i$ of type $t$ is given by
  $f_{t_i} \cdot \theta_{t,v,p}$, where:

\begin{itemize}
  \indentitem\item[$f_{t_i}$:]
    A value associated with every transition instance $t_i$, intuitively
      corresponding to how ``far'' the endpoints of a mutation are.
  \indentitem\item[$\theta_{t,v,p}$:]
    A learned cost for taking a transition of type $t$, if the source
    of the edge is in a validity state of $v$ and the word being mutated
    has polarity $p$.
\end{itemize}

The notation for $f_{t_i}$ is chosen to evoke an analogy to features,
  and we will refer to this quantity as the feature value.
We set $f_{t_i}$ to be 1\ in most cases;
  the exception are the edges over the WordNet hypernym tree,
  and the nearest neighbors edges.
In the first case, we take $\uparrow_{w \rightarrow w'}$ as transitioning
  from word $w$ to its hypernym $w'$ (and visa versa for
  $\downarrow_{w \rightarrow w'}$), and set:

\begin{align*}
  f_{\uparrow_{w \rightarrow w'}}   &= \log \frac{p(w')}{p(w)} &= \log p(w') - \log p(w) \\
  f_{\downarrow_{w \rightarrow w'}} &= \log \frac{p(w)}{p(w')} &= \log p(w) - \log p(w')
\end{align*}

We define $p(w)$ to be the probability (normalized frequency)
  of a word or any of its hyponyms in the Google N-Grams corpus
  \cite{key:2006brants-ngrams}.
Intuitively, this ensures that relatively long paths through fine-grained
  sections of WordNet are not unduly penalized.
For instance, the path from \textit{cat} to \textit{animal} traverses
  six intermediate nodes, \naive ly yielding a prohibitive
  search depth of 6.

For nearest neighbors edges, we take Neural Network embeddings learned
  in \newcite{key:huang-vectors} corresponding to each vocabulary entry.
We then define $f_{NN_{w \rightarrow w'}}$
  to be the arccosine of the cosine similarity between word vectors
  associated with lexical items $w$ and $w'$:

\begin{equation*}
  f_{NN_{w \rightarrow w'}}
    = \textrm{arccos} \left( \frac{w \cdot w'}{\|w\| \|w'\|} \right)
\end{equation*}

The set of features which fire along a path
  can be expressed as a vector $\bef$.
Each element of $\bef$ corresponds to sum of the values of that
  feature along the path.
Correspondingly, we define the weight vector $\btheta$ as the
  weight for every element of $\bef$.
The cost of a path can then be expressed as the dot product:
  $\theta^{\textrm{T}}\bef$.

%
% Generalize JC
%
\Subsection{inference-jc}{Generalizing Similarities}
An elegant property of our definitions of $f_{t_i}$ is its ability to
  generalize JC similarity, and upper-bound distributional similarity.
Let us assume we have words $w_1$ and $w_2$, with a least common subsumer $\textrm{lcs}$.
The JC distance $\textrm{dist}_{\textrm{jc}}(w_1, w_2)$ is:

\begin{equation}
\textrm{dist}_{\textrm{jc}}(w_1, w_2)
  = \log\frac{p(\textrm{lcs})^2}{p(w_1)p(w_2)}
\label{eqn:jc}
\end{equation}

For simplicity, we simplify $\theta_{\uparrow,v,p}$ and $\theta_{\uparrow,v,p}$
  as simply $\theta_\uparrow$ and $\theta_\downarrow$.
The derivation generalizes trivially to the the case where weights are
  further parameterized.
Without loss of generality, we also assume that a path in our search
  is only modifying a single word $w_1$, ending at a mutation of the
  word $w_2$.

We can factorize the cost of a path, $\btheta^{T}\bef$, along the path
  from $w_1$ to $w_2$ through its lowest common subsumer (lcs),
  $[w_1, w_1^{(1)}, \dots, \textrm{lcs}, \dots,  w_2^{(1)}, w_2]$,
  as follows:

\begin{align*}
\theta^{\textrm{T}}\phi
  &= \theta_\uparrow \left( 
    \left[\log p(w_1^{(1)}) - \log p(w_1)\right] +
    \dots
%   + \left[\log p(w_1^{(n)}) - \log p(\textrm{lcs})\right]
    \right) + \\
  &~~~~~~ \theta_\downarrow \left( 
    \left[\log p(\textrm{lcs}) - \log p(w_1^{(n)}) \right] +
    \dots
%    + \left[\log p(w_1) - \log p(w_1^{(1)})\right]
    \right) \\
  &= \theta_\uparrow \left( \log \frac{p(\textrm{lcs})}{p(w_1)} \right) +
     \theta_\downarrow \left( \log \frac{p(\textrm{lcs})}{p(w_2)} \right) \\
  &= \log \frac{ p(\textrm{lcs})^{\theta_\uparrow + \theta_\downarrow} }
               { p(w_1)^{\theta_\uparrow} + p(w_2)^{\theta_\downarrow} }
\end{align*}

Note that setting both $\theta_\uparrow$ and $\theta_\downarrow$ to 1 exactly
  yield the Formula \refeqn{jc} for JC distance.

It is also worth noting that the nearest neighbors path provides an
  upper bound on the true similarity between the start and end words
  in the path.
In this way, the search based approach presented here can be thought
  of as generalizing and formalizing the intuition that similar objects 
  have similar properties (e.g., as presented in
  \newcite{key:2013angeli-truth}) using both common classes of similarity
  metrics.

%
% PROBABILITY
%
\Subsection{inference-prob}{Confidence Estimation}
The last component in inference is translating a search path into a
  probability of truth.
We notice from \refsec{inference-transitions} that the \textit{cost}
  of a path can be represented as $\btheta^{T} \bef$.
We can normalize this value by negating every element of the weight
  vector and passing it through a sigmoid:

\begin{equation*}
\textrm{confidence} = \frac{1}{1 + e^{\btheta^{T} \bef}}
\end{equation*}

Importantly, note that the cost vector must be non-negative for the
  search to be well-defined, and therefore the confidence value will
  be constrained to be between 0 and $\frac{1}{2}$.

At this point, we have a confidence that the given path has not violated
  strict Natural Logic.
However, to translate this value into a probability
  we need to incorporate whether the inference path is
  confidently valid, or confidently invalid.
To illustrate, a fact with a low confidence should translate to a
  probability of $\frac{1}{2}$, rather than a probability of 0.

We therefore define the probability of validity as follows.
We take $v$ to be 1 if the final fact of our derivation is in the
  \textit{valid} state with respect to the antecedent,
  and -1 if this fact is in the \textit{invalid} state.
In practice, as our search is reversed, we take the state of the
  antecedent in our database when compared to the consequent, rather
  than visa versa.
For completeness, if no path is given we can set $v=0$.
The probability of validity then becomes:

\begin{equation}
  p(\textrm{valid}) = \frac{v}{2} + \frac{1}{1 + e^{v \btheta^{T} \bef}}
  \label{eqn:prob}
\end{equation}

Note that in the case where $v=-1$, the above expression reduces to
  $\frac{1}{2} - \textrm{confidence}$; in the case where $v=0$ it
  reduces to simply $\frac{1}{2}$.
Furthermore, note that the probability of truth makes use of the same
  parameters as the cost in the search.
Thus, as better weights are learned, the search is likewise more likely
  to produce derivations which would confidently support or disprove the
  query.
We proceed to describe how these weights are learned.
