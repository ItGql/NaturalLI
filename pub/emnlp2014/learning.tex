\Section{learning}{Learning Transition Weights}
The learning task can be viewed as a constrained optimization problem.
Subject to the constraint that all elements of the cost vector \btheta\ 
  must be non-negative, we optimize the probability from
  Equation \refeqn{prob} compared against the gold annotation.
As training data, we are given a number of facts, annotated with a
  truth value of \textit{true} or \textit{false}.
We assume that all facts in our database are true; therefore,
  $p(\textrm{valid})$ corresponds directly to $p(\textrm{true})$.

We learn costs using an iterative algorithm.
At each iteration, we take the costs from the previous iteration
  and run the derivation search over every example, providing
  a predicted probability of truth for each query fact.
Optimizing the likelihood of the training data according to
  Equation \refeqn{prob} is impractical as the objective is nonconvex;
  the rest of today is intended to be spent on implementing a piecewise
  optimization to mitigate this.
A simple heuristic (weight = 1 - \# times this feature fired in good examples
  / \# of times feature fired total) does ok, but is a little braindead.

%In particular, we construct a dataset of $(x,y)$ pairs, where $x$ is the
%  featurized path for a particular example, and $y$ is whether the
%  path ended in a correct state.
%A correct state is defined trivially between \textit{valid} and
%  \textit{invalid}.
%In cases where the gold annotation contains a state of 
%  \textit{unknown validity}, any prediction is marked incorrect.
%
%The resulting weights -- in one-to-one correspondence with the costs
%  for our search -- are then negated and normalized via a bilinear
%  transform to fall between 0 and $-5$, with a mean of $-1$.
%A default weight of $-2$ is assigned for any feature which did not
%  appear in the training data, excepting the case where a previous
%  iteration of the algorithm had set a value for that weight, in
%  which case the previous value is used.
