\Section{maccartney}{MacCartney's Natural Logic}
% Motivate Natural Logic
Broadly, Natural Logic aims to capture common logical
  inferences by appealing directly to the structure of language,
  as opposed to running deduction on an abstract logical form.
Natural Logic is inspired largely by the Aristotelian
  syllogisms and subsequent traditional logics
  \cite{key:2008vanbenthem-natlog}.
While not all inferences in first-order logic are in Natural Logic,
  it nonetheless allows for a wide range of intuitive inferences in
  a computationally efficient manner.
%That is to say, Natural Logic takes the meaning representation of a
%  sentence to be the surface form of the sentence itself.
%In part, this lends itself to computationally efficient inference;
%  in another part, it frees the system from parsing to an abstract
%  logic -- this is particularly relevant in our case, where the set
%  of potential antecedents is very large.

% Introduce MacCartney + Related Work
We build off of the variant of Natural Logic introduced by
  the NatLog inference system (MacCartney and Manning, 2007; 2008; 2009),
  \nocite{key:2007maccartney-natlog}
  \nocite{key:2008maccartney-natlog}
  \nocite{key:2009maccartney-natlog}
  based off of earlier theoretical work on Natural Logic and 
  Monotonicity Calculus
  \cite{key:1986benthem-natlog,key:1991valencia-natlog}.
Later work has formalized many aspects of the logic
  \cite{key:2012icard-natlog,key:2013djalali-natlog};
  formally we are adopting the semantics of outlined in
  \newcite{key:2014icard-natlog}.
%  which we appeal to but are not necessary for a 
%  appeal to in future sections, but are not necessary for an understanding
%  of the contributions of this work.
%An elegant introduction to Natural Logic can be found in
%  \newcite{key:2014icard-natlog}; a thorough treatment of
%  MacCartney's Natural Logic can be found in \newcite{key:2009maccartney-natlog}.

At a high level, Natural Logic proofs operate by mutating spans of text
  in precise ways to ensure that the mutated sentence follows from the
  original -- each step is much like a syllogistic inference.
We therefore must introduce three components for a complete proof system:
  we define the valid atomic relations between lexical entries
  (\refsec{maccartney-relations}), the effect these mutations have
  on the validity of the inference (\refsec{maccartney-polarity}),
  and a practical system for executing these proofs.
We introduce MacCartney's alignment-based approach in
  \refsec{maccartney-proof}, and show that we can generalize and
  simplify this system in \refsec{search}.

%
% MacCartney Relations
%
\Subsection{maccartney-relations}{Lexical Relations}

% -- Relations Figure --
\begin{figure}[t]
\begin{center}
  \resizebox{0.48\textwidth}{!}{
    \begin{tabular}{ccc}
      \equivalentVenn & \forwardVenn & \reverseVenn \\
      \negateVenn & \alternateVenn & \coverVenn
    \end{tabular}
  }
\end{center}
\caption{
  The model-theoretic interpretation of the MacCartney relations.
  The figure shows the relation between the denotation of
    $\varphi$, in dark gray, compared to the denotation of
    of $\psi$, in light gray.
  The universe is denoted by $\sD$.
  \label{fig:relations}
}
\end{figure}
% --                 --

\newcite{key:2007maccartney-natlog} introduce seven set-theoretic
  relations between the denotations of any two lexical items.
The denotation of a lexical item is the set of objects in the domain
  of discourse $\sD$ which that lexical item refers to.
For instance, the denotation of \w{cat} would be the set of all cats.
The denotations of two lexical items can then be compared in terms of
  set algebra: if we define the set of cats to be $\varphi$ and
  the set of animals to be $\psi$, we can state that
  $\varphi \subseteq \psi$.
%This example should evoke a taste of Natural Logic: e.g., if
%  $\varphi \subseteq \psi$, then anything which holds for all elements
%  in $\psi$ (all animals) must hold for all elements of $\varphi$
%  (all cats).

The six informative relations are summarized in \reffig{relations};
  a seventh relation (\independent) corresponds to to the completely
  uninformative relation.
For instance, the example search path in \reffig{teaser} makes use of
  the following relations:

\begin{center}
\begin{tabular}{rcl}
\w{No $x$ $y$} & \negate     & \w{The $x$ $y$} \\
\w{cat}        & \reverse    & \w{carnivore} \\
\w{animals}    & \reverse    & \w{mouse} \\
\w{mouse}      & \equivalent & \w{a mouse} \\
\end{tabular}
\end{center}

%The middle two middle entries have straightforward interpretations:
%  the set of cats is a subset of the set of carnivores; the set of
%  mice is a subset of animals.
%The last entry states that the denotation of \w{mouse} and
%  \w{a mouse} is close enough that they can be considered equivalent.

Denotations are not required to be in the space of entities.
In the first example, the denotations of \textit{No} and \textit{The}
  are in the space of quantifiers
  \mbox{$e \rightarrow (e \rightarrow t)$}:
  functions from entities $e$ to truth values $t$.
%The first entry is somewhat subtle in that the denotation of quantifiers
%  is not over the universe of entities $e$, but rather over functions
%  from entities to truth values $t$:
Our claim is really the conjunction of two claims:
  $\forall x \forall y ~ \lnot\left(\w{no}~x~y \land \w{the}~x~y \right)$
  and
  $\forall x \forall y ~ \left(\w{no}~x~y \lor \w{the}~x~y\right)$.
This is analogous to the formal construction of the
  set-theoretic definition of \negate\ in \reffig{relations}:
  $\varphi \cap \psi = \varnothing$ and $\varphi \cup \psi = \sD$
  (see \newcite{key:2014icard-natlog}).
%A more thorough treatment can be found in
%  \newcite{key:2014icard-natlog}.

%Note that \forward\ and \reverse\ are inverses.
Examples of the last two relations (\alternate\ and \cover)
and the complete independence relation (\independent) are given below:

\begin{center}
\begin{tabular}{rcl}
\w{cat}        & \alternate   & \w{dog} \\
\w{animal}     & \cover       & \w{nonhuman} \\
\w{cat}        & \independent & \w{friendly} \\
\end{tabular}
\end{center}

%We proceed to describe the relationship between these lexical mutations,
%  and the validity of executing the mutation in a derivation.

%
% Monotonicity & Polarity
%
\Subsection{maccartney-polarity}{Monotonicity and Polarity}
% Motivate
We can determine the relation relation between 
  lexical items from the previous section;
  however, we still need a theory for how to propagate the relation
  induced by a lexical mutation ``up the parse tree'' to become
  a relation between two sentences.
For example, \textit{cat} \forward\ \textit{animal},
  and \textit{some cat meows} \forward\ \textit{some animal meows},
  but
  \textit{no cat barks} \reverse\ \textit{no animal barks}.
Despite differing by the same lexical relation,
  the first example is entailed, while the second is not.

% Introduce Monotonicity
We appeal two important concepts: \textit{monotonicity} as a
  property of arguments to natural language quantifiers,
  and \textit{polarity}
  as a property of lexical items in a sentence.
Much like monotone functions in calculus,
  an [upwards] monotone quantifier has an output truth value which is
  non-decreasing (i.e., material implication; true cannot become false)
  as the input ``increases'' (i.e., in terms of set containment).
From the example above, \textit{some} is upwards monotone in its first
  argument, and \textit{no} is downwards monotone in its first argument.

%% Explain monotone
%To offer a concrete example, \textit{some} is a monotone quantifier taking two
%  entities as input
%  (i.e., \w{some $x$ $y$} -- \w{some cats eat mice}),
%  and returning a truth value.
%Therefore, we can replace an argument to \w{some} with a superset of that
%  argument, and be guaranteed that the truth value of the new
%  predicate will be ``at least'' the truth value of the original statement.
%The ``at least'' relation on truth values is trivially defined
%  as $F \leq T$; therefore, for two truth values $t_1$ and $t_2$,
%  $t_1 \leq t_2$ is precisely material implication $t_1 \Rightarrow t_2$.
%With this we can now show that, e.g., 
%  \w{some cats eat mice} $\Rightarrow$ \w{some animals eat mice}.
%
%% Whirlwind through antitone + nonmonotone
%Analogous to monotone quantifiers, we can define \textit{antitone}\footnote{
%    Monotone and antitone are often referred to as
%    \textit{monotone up} and \textit{monotone down}.
%  }
%  quantifiers as quantifiers which have an output truth value which is
%  non-increasing as the input increases.
%For instance, \w{no} is an antitone quantifier.
%Note that quantifiers can have different monotonicity for each argument:
%  \w{all} is monotone in its first and antitone in its second argument.
%Furthermore, not all quantifiers are monotone or antitone --
%  \w{most} is the classic example of a quantifier which is
%  \textit{nonmonotone} in its first argument.

% Polarity
%So far we have looked only at sentences with a single quantifier;
%  moreover, monotonicity itself is not always sufficient to warrant
%  mutations on lexical items.
\textit{Polarity} is
  a property of lexical items in a sentence determined by the
  quantifiers acting on it.
All lexical items have \textit{upward} polarity by default;
  upwards monotone quantifiers preserve polarity,
  and downwards monotone quantifiers reverse polarity.
For example, \w{cats} in \w{all cats eat mice} has downward polarity; or,
  \w{mice} in \w{no cats don't eat mice} has upward polarity
  (in the scope of two downward monotone quantifiers).

% -- Projection Table --
% I'm assuming the context is additive and multiplicative
% See Lemma 2.4 (p.11) of http://link.springer.com/article/10.1007%2Fs11225-012-9425-8
\begin{table}[t]
	\begin{center}
	\begin{tabular}{c|cc}
    \textbf{Relation} & \multicolumn{2}{|c}{\textbf{Polarity of Context}} \\
             & Upward & Downward \\
    \hline
    $e_1 \forward     e_2$ & $s_1 \forward     s_2$ & $s_1 \reverse     s_2$ \\ 
    $e_1 \reverse     e_2$ & $s_1 \reverse     s_2$ & $s_1 \forward     s_2$ \\ 
    $e_1 \alternate   e_2$ & $s_1 \alternate   s_2$ & $s_1 \cover       s_2$ \\ 
    $e_1 \cover       e_2$ & $s_1 \cover       s_2$ & $s_1 \alternate   s_2$ \\ 
    $e_1 \equivalent  e_2$ & $s_1 \equivalent  s_2$ & $s_1 \equivalent  s_2$ \\ 
    $e_1 \negate      e_2$ & $s_1 \negate      s_2$ & $s_1 \negate      s_2$ \\ 
    $e_1 \independent e_2$ & $s_1 \independent s_2$ & $s_1 \independent s_2$
	\end{tabular}
	%(caption)
	\caption{
    The projection table used for a relation between a phrase $e_1$ and
      its candidate mutation $e_2$, in terms of the produced relation
      between sentence $s_1$ and the new sentence $s_2$.
    Values are given for when the entities are in a monotone and
      antitone context.
		\label{tab:projectivity}
	}
	\end{center}
\end{table}

% Projection
The relation between two sentences differing by a single lexical
  relation is then given in the projection table:
  \reftab{projectivity}.\footnote{
    Note that this table optimistically assumes every quantifier is
    \textit{additive} and \textit{multiplicative}, as defined
    in \newcite{key:2012icard-natlog}.
  }


%We now have a repertoire or lexical mutations, and a polarity marking
%  on each lexical item.
%We have two remaining elements to define for a full proof theory:
%  how do lexical mutations of a word with a certain polarity
%  \textit{trickle up} the sentence to affect the relation between
%  entire sentences, and how do we chain individual lexical mutations
%  together for a full derivation.
%The first is studied in depth by \newcite{key:2012icard-natlog};\footnote{
%    Formally, we optimistically assume every quantifier 
%    is \textit{additive} and \textit{multiplicative}.
%  }
%  we axiomatically define the relation between two sentences differing
%  by a mutation in \reftab{projectivity}.
%The second is discussed in the next section.

%
% Inference as Alignment
%
\Subsection{maccartney-proof}{Proof By Alignment}

% -- Join Table--
\begin{table}[t]
	\begin{center}
	\begin{tabular}{|c||c|c|c|c|c|}
    \hline
    $\bowtie$ & $\forward$ & $\reverse$ & $\negate$ & $\alternate$ & $\cover$ \\
    \hline
    $\forward$ & $\forward$ & $\independent$ & $\alternate$ & $\alternate$ & $\independent$ \\
    $\reverse$ & $\independent$ & $\reverse$ & $\cover$ & $\independent$ & $\cover$ \\
    $\negate$ & $\cover$ & $\alternate$ & $\equivalent$ & $\reverse$ & $\forward$ \\
    $\alternate$ & $\independent$ & $\alternate$ & $\forward$ & $\independent$ & $\forward$ \\
    $\cover$ & $\cover$ & $\independent$ & $\reverse$ & $\reverse$ & $\independent$ \\
    \hline
	\end{tabular}
	%(caption)
	\caption{
    The join table, as copied from \newcite{key:2012icard-natlog}.
    Note that the $\independent$ always joins to yield $\independent$,
    and $\equivalent$ always joins to yield the input relation.
		\label{tab:join}
	}
	\end{center}
\end{table}

\newcite{key:2007maccartney-natlog} approach the inference task in
  the context of inferring whether a single relevant premise entails
  a query.
%In the framework of \newcite{key:2007maccartney-natlog}, the task is
%  to infer whether a single logical antecedent entails a query
%  consequent.
Their approach is to generate an alignment between the premise
  and the query, and classify each aligned segment into one of
  the relations in \reffig{relations}.
Inference then reduces to projecting each of these relations
  according to \reftab{projectivity} and iteratively \textit{joining} 
  two projected relations together to get the final entailment
  relation.
This \textit{join table}, denoted as $\bowtie$, 
  is given in \reftab{join}.

To illustrate we can consider MacCartney's example inference from
  \w{Stimpy is a cat} to \w{Stimpy is not a poodle}.
An alignment of the two statements would provide three lexical
  mutations:
    \mbox{$r_1\coloneqq\w{cat}\rightarrow\w{dog}$}, 
    \mbox{$r_2\coloneqq\cdot\rightarrow\w{not}$},
    and \mbox{$r_3\coloneqq\w{dog}\rightarrow\w{poodle}$}.
The initial relation $r_0$ is axiomatically \equivalent.
Each of these then project through the projection function $\rho$
  from \reftab{projectivity},
  and are joined by the join relation:

\begin{equation*}
  r_0 \bowtie \rho(r_1) \bowtie \rho(r_2) \bowtie \rho(r_3)
\end{equation*}

In MacCartney's work this style of proof is presented as a table,
  as follows.
The last column ($s_i$) is the relation between the premise and the
  $i^{th}$ step in the proof, and is constructed inductively as
  $s_i \coloneqq s_{i-1} \bowtie \rho(r_i)$:

\begin{center}
\begin{tabular}{rl|ccc}
  \multicolumn{2}{c|}{Mutation} & $r_i$ & $\rho(r_i)$ & $s_i$ \\
  \hline
  $r_1$ & \w{cat}$\rightarrow$\w{dog}    & \alternate & \alternate & \alternate \\
  $r_2$ & $\cdot\rightarrow$\w{not}      & \negate    & \negate    & \forward \\
  $r_3$ & \w{dog}$\rightarrow$\w{poodle} & \reverse   & \forward   & \forward \\
\end{tabular}
\end{center}

%\noindent where $x_i$ is the surface form of the fact at inference step
%  $i$, $\beta(e_i)$ is the lexical relation, as per
%  \reffig{relations}, $\beta(x_{i-1}, e_i)$ is the relation projected
%  according to \reftab{projectivity}, and $\beta(x_0, x_i)$ is
%  the relation between the antecedent and the current candidate fact.
%$\beta(x_0, x_i)$ is obtained from repeatedly applying the join table
%  in \reftab{join} to $\beta(x_{0}, x_{i-1})$ and $\beta(x_{i-1}, e_i)$.
%For instance, joining $\beta(x_1, x_1)$ with $\beta(x_1, e_2)$
%  (\alternate\ $\bowtie$ \negate) yields $\beta(x_1, x_2)$ (\forward).
%
%The final relation between the antecedent and the consequent is taken
%  to be the last $\beta(x_0, x_i)$ entry.

In our example, we would conclude that
  \w{Stimpy is a cat} \forward\ \w{Stimpy is not a poodle}
  since $s_3$ is \forward,
  and therefore the inference is valid.
