\Section{related}{Related Work}
\todo{don't plagiarize the EMNLP2014 paper}
\todo{put in Aristo papers}
% -- Extending OpenIE
% Richard
\newcite{key:2013chen-completion} and \newcite{key:2013socher-completion}
  use Neural Tensor Networks to predict unseen relation triples in
  WordNet and Freebase, following a line of work by
  \newcite{key:2011bordes-completion} and
  \newcite{key:2012jenatton-completion}.
% Universal schemas
\newcite{key:2012yao-schemas} and \newcite{key:2013riedel-schemas}
  present a related line of work, inferring new relations between
  Freebase entities via inference over both Freebase and
  OpenIE relations.
In contrast, this work runs inference over arbitrary text, without 
  restricting itself to a particular set of relations, or even entities.
%This work, however, focuses primarily on common-sense reasoning rather
%  than inferring relations between named entities.

% -- RTE
This work is similar in many ways to work on 
  recognizing textual entailment -- e.g., 
  \newcite{key:2010-schoenmackers-horn}, \newcite{key:2011berant-entailment}.
Work by \newcite{key:2013lewis-entailment} is particularly relevant,
  as they likewise evaluate on the FraCaS suite (Section 1;
  89\% accuracy with gold trees).
They approach entailment by constructing a CCG parse of the query,
  while mapping questions which are paraphrases of each other to the
  same logical form using distributional relation clustering.
However, their system is unlikely to scale to either our large
  database of premises, or our breadth of relations.

% Paraphrase-based Q/A
\newcite{key:2014fader-openqa} propose a system for question answering
  based on a sequence of paraphrase rewrites followed by a fuzzy query to
  a structured knowledge base.



