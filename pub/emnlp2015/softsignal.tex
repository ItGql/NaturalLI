\Section{softsignal}{An Evaluation Function for NaturalLI}
% Natural Logic doesn't get everything
There are many cases -- particularly as the length of the premise and the hypothesis grow --
  where despite our improvements NaturalLI will fail to find any supporting
  premises; for example:
%This can be from a necessity to do complex reasoning (as in the example from the 
%  introduction), or simply complex nonlocal reasoning, as in:

% Here's something simple it struggles with
\entailmentExample
{Food serves mainly for growth, energy and body repair, maintenance and protection.}
{Animals get energy for growth and repair from food.}

% Why does it struggle -- and why it should
In addition to requiring reasoning with multiple implicit premises (a concomitant weak
  point of natural logic), a correct interpretation of the sentence requires
  fairly nontrivial nonlocal reasoning: \w{Food serves mainly for $x$}
  $\rightarrow$ \w{Animals get $x$ from food}.

% Why a lexical classifier would do well
%Nonetheless, objectively this is not a difficult case of entailment.
Nonetheless, there are plenty of clues that the two sentences are related, and even a simple
  entailment classifier would get the example correct.
We build such a classifier and adapt it as an evaluation function
  inside NaturalLI in case no premises are found during search.
%  (\refsec{softsignal-classifier}) built only
%  on dense lexical overlap features, and adapt it as a backoff inside NaturalLI
%  in case no premises are found during search.
%  then show how NaturalLI can be used to provide soft signals informing the classifier.

%% Why are soft signals valuable
%The transition types in NaturalLI's search are roughly the analogue of complex features
%  used in entailment systems, e.g., for the RTE challenges.
%WordNet features are encoded in the hypernym and antonym edge types; distributional
%  similarity in the nearest neighbors edges; negation scope in the quantifier
%  mutations.
%We argue that our approach has two advantages here: 
%  first, it allows our classifier to implement a clean, unlexicalized set of
%  features while retaining the expressive power of these more powerful features.
%Second, and more importantly, NaturalLI is still allowed to find its own supporting
%  (or negating) premises independent of the classifier.
%The classifier can in this sense be thought of as a ``backoff'' for NaturalLI,
%  offering some answer when NaturalLI could not provide one.


%
% Soft Signals
%
\Subsection{softsignal-classifier}{A Standalone Entailment Classifier}
% high-level overview
Our entailment classifier is designed to be as domain independent as possible;
  therefore we define only 5 unlexicalized real-valued features, with an 
  optional sixth feature encoding the score output by the Solr
  information extraction system (in turn built upon Lucene).
% Why simple feature set
Additional features were not shown to improve performance on the training set.
In fact, this classifier is a stronger baseline than it may seem: evaluating
  the system on RTE-3 \cite{key:2007giampiccolo-rte} yielded \textbf{63.75}\% accuracy --
  2 points above the median submission.
%Additional features were not shown to substantially improve performance, and
%  were therefore dropped for simplicity.
%Features tried included the BLEU score between the phrases, the overlap between
%  words in the premise and hypothesis, the overlap between words sharing a 
%  part of speech tag in the premise and hypothesis, and the length difference
%  between the phrases.

All five of the core features are based on an alignment of keyphrases between the
  premise and the hypothesis.
% Define a keyphrase
A keyphrase is defined as a span of text which is either
  (1) a possibly empty sequence of adjectives and adverbs followed by a 
      sequence of nouns, and optionally followed by either \w{of} or the possessive
      marker (\w{'s}), and another noun (e.g., \w{sneaky kitten} or \w{pail of water});
  (2) a possibly empty sequence of adverbs followed by a verb (e.g.,
      \w{quietly pounce}); or
  (3) a gerund followed by a noun (e.g., \w{flowing water}).
The verb \w{to be} is never a keyphrase.
We make a distinction between a \textit{keyphrase} and a \textit{keyword} --
  the latter is a single noun, adjective, or verb.

\begin{figure*}
\begin{center}
\begin{tikzpicture}[scale=0.75, every node/.style={scale=0.75}]
\tikzset{
  arm angleA/.initial={0},
  arm angleB/.initial={0},
  arm lengthA/.initial={0mm},
  arm lengthB/.initial={0mm},
  arm length/.style={%
    arm lengthA=#1,
    arm lengthB=#1,
  },
  arm/.style={
    to path={%
      (\tikztostart) -- ++(\pgfkeysvalueof{/tikz/arm angleA}:\pgfkeysvalueof{/tikz/arm lengthA}) -- ($(\tikztotarget)+(\pgfkeysvalueof{/tikz/arm angleB}:\pgfkeysvalueof{/tikz/arm lengthB})$) -- (\tikztotarget)
    }
  },
}
\matrix[column sep=0.0em,row sep=0.5cm,matrix of nodes,row 2/.style={coordinate}] (m) {
  |(he)|\darkred{Heat energy} & is being & |(transferred)|\darkred{transferred} & when & 
    |(stove)|a \darkred{stove} is & |(used)|\darkred{used} & |(boil)|to \darkred{boil} & 
    |(water)|\darkred{water} & |(pan)|in a \darkred{pan}. \\
\\
  When you & |(hw)|\darkred{heat water} & on a & |(stove2)|\darkred{stove}, & 
    |(te)|\darkred{thermal energy} & 
    is & |(transferred2)|\darkred{transferred}. \\
};
\begin{scope}[every path/.style={line width=4pt,white,double=black},every to/.style={arm}, arm angleA=-90, arm angleB=90, arm length=5mm]
  \draw (water) to (hw);
  \draw (he) to (te);
  \draw (transferred) to (transferred2);
  \draw (stove) to (stove2);
\end{scope}
\end{tikzpicture}
\end{center}
\caption{
\label{fig:alignment}
An illustration of an alignment between a premise and a hypothesis. 
Keyphrases can be multiple words (e.g., \w{heat energy}),
  and can be approximately matched (e.g., to \w{thermal energy}).
In the premise, \w{used}, \w{boil} and \w{pan} are unaligned.
Note that \w{heat water} is incorrectly tagged as a compound noun.
}
\end{figure*}

% Alignment algorithm
We then align keyphrases in the premise and hypotheses by applying a series of sieves.
First, all exact matches are aligned to each other.
Then, prefix or suffix matches are aligned, then if either keyphrase contains
  the other they are aligned as well.
Last, we align a keyphrase in the premise $p_i$ to a keyphrase in the hypothesis
  $h_k$ if there is an alignment between $p_{i-1}$ and $h_{k-1}$ and between
  $p_{i+1}$ and $h_{k+1}$.
This forces any keyphrase pair which is ``sandwiched'' between aligned pairs to be
  aligned as well.
An example alignment is given in \reffig{alignment}.

% Features
%The classifier features are then statistics on this generated alignment.
Features are extracted for the number of alignments, the numbers of alignments
  which do and do not match perfectly, 
  and the number of keyphrases in the premise and hypothesis
  which were not aligned.
A feature for the Solr score of the premise given the hypothesis is optionally
  included; we revisit this issue in the evaluation.

%
% Soft Signals from NaturalLI
%
\Subsection{softsignal-keywords}{An Evaluation Function for Search}
% Intuition as value function
A version of the classifier constructed in 
  \refsec{softsignal-classifier}, but over \textit{keywords} rather
  than keyphrases can be incorporated directly into
  NaturalLI's search to give a score for each candidate premise visited.
This can be thought of as analogous to the evaluation function in game-playing
  algorithms.
%  -- even though an agent cannot play a game of Chess to completion,
%  at some depth it can apply an evaluation function to its leaf states.

%Unlike the keyphrase spans in the classifier from \refsec{softsignals-classifier},
%  keywords are single tokens in NaturalLI -- note that this is not always
%  a single word (e.g., \w{fire truck} is a single token.
%However, this nonetheless limits the number of fuzzy alignments we 
%  are able to make.
%We can make use of the classifier we constructed in 
%  \refsec{softsignal-classifier} to score each candidate premise during
%  the search in NaturalLI against a set of known premises.

% Can do this incrementally
Using keywords rather than keyphrases is in general a hindrance to the
  fuzzy alignments the system can produce.
Importantly though, this allows the feature values to be computed 
  incrementally as the search
  progresses, based on the score of the parent state and the mutation or
  deletion being performed.
For instance, if we are deleting a word which was previously aligned perfectly
  to the premise, we would subtract the weight for a perfect and imperfect
  alignment, and add the weight for an unaligned premise keyphrase.

% Soft negation
In addition to finding entailments from candidate premises, our system also
  allows us to encode a notion of likely negation.
%Note that in the context of NaturalLI, we can define a more precise notion of
%  overlap, which will be useful for capturing a notion of likely logical
%  negation.
We can consider the following two statements \naive ly sharing every keyword.
Each token marked with its polarity:
%To illustrate, \naive ly the following two statements share every keyword:

\entailmentExample
{\tagUp{some} \tagUp{cats} \tagUp{have} \tagUp{tails}}
{\tagUp{no} \tagDown{cats} \tagDown{have} \tagDown{tails}}

However, we note that all of the keyword pairs are in opposite polarity contexts.
We can therefore define a pair of keywords as \textit{matching} in NaturalLI
  if the following two conditions hold: 
  (1) their lemmatized surface forms match exactly, and 
  (2) they have the same polarity in the sentence.
The second constraint encodes a good approximation for negation.
To illustrate, consider the polarity signatures of common operators:

\begin{center}
\begin{tabular}{lcc}
%  \hline
  \textbf{Operators} & \textbf{Subj. polarity} & \textbf{Obj. polarity} \\
%  \hline
  \w{Some, few, etc.}    & $\uparrow$   & $\uparrow$ \\
  \w{All, every, etc.}     & $\downarrow$ & $\uparrow$ \\
  \w{Not all, etc.} & $\uparrow$   & $\downarrow$ \\
  \w{No, not, etc.}      & $\downarrow$ & $\downarrow$ \\
  \w{Most, many, etc.}    & --           & $\uparrow$ \\
  %\hline
\end{tabular}
\end{center}
  
We note that, e.g., \textit{some}
  and \textit{no} induce the exact opposite polarity on their arguments;
  similarly \textit{all} and \textit{not all} have opposite polarities.
%The conspicuous counterexamples to this is the operator pairs \w{all} and \w{no},
%  which have the same monotonicity in their subjects but are nonetheless negations
%  of each other.
Otherwise, pairs of operators which share half their signature are 
  nearly always compatible with each other (e.g., \w{some} and \w{all}).
%In all, we consider this a good simple approximation at a low cost.

This suggests a natural criterion for likely negation:
  If the highest classifier score is produced by a contradictory candidate 
  premise, we have reason to believe that we 
  may have found a contradiction.
To illustrate with our example, NaturalLI would mutate \w{no cats have tails}
  to \w{the cats have tails}, at which point it has found a 
  contradictory candidate premise which has perfect overlap with the 
  premise \w{some cats have tails}.
Even had we not found the exact premise, this would suggest that
  the hypothesis is likely contradicted.


%This leads naturally into the second soft signal that NaturalLI can convey, when
%  it believes the hypothesis is false but cannot prove it explicitly.
%We consider this the case when we get a higher overlap score from a false
%  candidate premise than we did from any true one.
%Considering our example from earlier, a search from the hypothesis
%  \w{no cats have tails} would have an overlap of zero with \w{some cats have tails}.
%But, as soon as NaturalLI negates the quantifier \w{no}, the polarity of both
%  \w{cats} and \w{tails} matches and the overlap becomes two.
%In this way, even if NaturalLI had not found the exact premise, it would have
%  strong reason to suspect that the hypothesis is wrong on the basis of a
%  false search state (i.e., candidate premise) having a higher alignment to 
%  the actual premise than a true search state.




%% How we incorporate it into search
%Updating the counts of the features in the classifier
%  is trivial during the search itself.
%At every transition, we are either mutating a word, or deleting a word.
%If an aligned word is mutated, the score is updated according to whether it
%  mutates to the exact aligned word, or mutates away from the exact aligned word.
%If the transition deleted a word, the score is updated according to whether
%  the word was 
%If a search state has a score $s_0$, and 
%
%% Segment feature space
%The feature set in \refsec{softsignals-classifier} can be segmented into two
%  types of features:
%  those which are constant throughout NaturalLI's search, and those which 
%  can change during search.
%In particular, the number of alignments, the number of keyphrases in the premise
%  which were not aligned, and the Lucene score of the premise cannot change during
%  search without recomputing the alignments.
%In contrast, the number of perfect and imperfect alignments, as well as the number
%  of unaligned keyphrases in the hypotheses can change during search.
%The former through mutations, and the latter through deletions.
%
%
%% Signal 1: match overlap
%In addition to explicitly confirming or contradicting an entailment, NaturalLI can
%  provide at least two soft signals for the backoff classifier.
%The first of these is providing a better calibrated exact alignment match measure.
%For example, if \w{cat} in the premise aligns to \w{furry cat} in hypothesis,
%  and NaturalLI finds finds that \w{furry} can be deleted during search, the
%  exact match count should go up by one.
%Symmetrically, the inexact match count should go down by one.
%
%% Semantics of a matching keyword
%This already allows us to capture notions of soft matching, distributional similarity
%  (via nearest neighbors edges), etc.
%%However, we should take care to be somewhat careful; to illustrate, 
%However, we should be somewhat cautious; to illustrate, 
%  \naive ly the following two statements share every keyword:
%
%\entailmentExample
%{some cats have tails}
%{no cats have tails}

%\begin{table}
%\begin{center}
%\end{center}
%\caption{\label{tab:operatormono}
%  The five common monotonicity configurations of an operator, and an exemplar
%    of that operator.
%  Operators which are the negation of each other never share a signature, and
%    most often have the exact opposite signature.
%}
%\end{table}

%We therefore define a pair of keywords as \textit{matching} if the following two
%  conditions hold: 
%  (1) their lemmatized surface forms match exactly, and 
%  (2) they have the same polarity in the sentence.
%The second constraint encodes a good approximation for negation (see \reftab{operatormono});
%  e.g., \textit{some}
%  and \textit{no} induce the exact opposite polarity on their arguments,
%  and therefore in our example above there is no keyword overlap.
%The conspicuous counterexamples to this is the operator pairs \w{all} and \w{no},
%  which have the same monotonicity in their subjects but are nonetheless negations
%  of each other.
%Otherwise, any pair of operators which share half their signature are at least
%  compatible with each other (e.g., \w{some} and \w{all}).
%In all, we consider this a good simple approximation at a low cost.

%% Implementation
%Both of these signals are incorporated straightforwardly into NaturalLI.
%In addition to the large knowledge base, a query is additionally augmented
%  with a small set of facts which are considered better candidates for
%  a valid premise (e.g., are returned by an IR system).
%A search state then carries around the alignment score to each premise,
%  and updates this score with each transition.
