\Section{results}{Evaluation}

\Subsection{fracas}{FraCaS}

\Subsection{aristo}{Aristo Regents Science Exams}
% What is Aristo?
We evaluate our entailment system on the Regents Science Exam portion of
  the Aristo dataset \cite{key:2013clark-aristo,key:2015clark-aristo}.
The dataset consists of a collection of multiple-choice science questions
  from the New York Regents 4$^{\textrm{th}}$ Grade Science Exams
  \cite{key:NYSED}.
This is then cast as an question answering task -- given a corpus of
  supporting facts, find evidence for the truth of a particular answer.

% Why do we use it?
Our system is in many ways well-suited to the dataset.
Although certainly many of the facts require complex reasoning
  (see \refsec{aristo-analysis}), the majority of questions can be
  answered from a single premise -- allowing Natural Logic to be used
  as the inference formalism.
Unlike FraCaS or the RTE challenges, however, the task does not have explicit
  premises to run inference from, but rather must infer the truth of the
  hypothesis from a large collection of supporting text.
In contrast to the common-sense reasoning task in \newcite{key:2014angeli-naturalli}
  though, the queries in the aristo dataset are relatively longer, complete sentences.
This necessitates the adaptations in \refsec{naturalli}, and benefits more from the
  soft signals from \refsec{softsignal}.

\Subsubsection{aristo-data}{Data Processing}
% Corpora
We make use of two collections of corpora for our experiments.
The first of these is the Barrons study guide (\textsc{Barrons}, 
  consisting of \num{1200} sentences.
This is the corpus used by \newcite{key:2015hixon-aristo} for their conversational
  dialog engine \knowbot, and therefore constitutes a more fair comparision against 
  their results.
However, we also make use of the full \textsc{Scitext} corpus \cite{key:2014clark-aristo}.
This corpus consists of \num{1316278} supporting sentences, 
  including the Barrons alongside simple Wikipedia, dictionaries, and a science
  textbook.

% Pre-processing
Since we lose all document context when searching over the corpus with NaturalLI,
  we first pre-process the corpus to resolve high-precision cases of
  pronominal coreference, via a set of simple high-precision sieves.
The corpus is then filtered for non-ascii characters (yielding \num{945956} sentences),
  and exact duplicate sentences are removed.
This results in a total of \num{822748} facts in the supporting corpus.

% Lucene
These sentences were then indexed using Solr.
The set of promising premises for the soft alignment in \refsec{softsignal}, as well as
  the Lucene score feature in the lexical classifier (\refsec{softsignal-classifier}),
  were obtained by querying Solr using the default similarity metric and scoring function.

% Questions to Candidates
On the query side, questions were converted to answers using the same methodology as
  \newcite{key:2015hixon-aristo}.
In cases where the question contained multiple sentences, only the last sentence
  was considered.
As discussed below, this renders these multi-sentence questions effectively out of
  scope for the model.

\Subsubsection{aristo-data}{Multiple-Choice as Entailment}

\Subsubsection{aristo-data}{Experimental Results}

%
% ARISTO NATURALLI ONLY
%
% Some definitions
\def\t#1{\small{#1}}
\def\b#1{\t{\textbf{#1}}}
\def\colspaceS{2.0mm}
\def\colspaceM{3.0mm}
\def\colspaceL{4.0mm}

% The table
\begin{table}
\begin{center}
\begin{tabular}{l@{\hskip \colspaceL}c}
\hline
\textbf{System} & \textbf{Accuracy} \\
\hline
\t{NaturalLI Only}                 & \t{54.6} \\
\t{Lucene Only}                    & \t{63.9} \\
\t{Alignment Features Only}        & \t{67.6} \\
\t{Alignment + Lucene}             & \t{72.2} \\
\t{Alignment + Lucene + NaturalLI} & \b{71.3} \\
\hline
\end{tabular}
\end{center}
% The caption
\caption{
\label{tab:aristonaturalli}
Accuracy of various systems on the Aristo science questions dataset.
\todo{numbers are on dev.}
}
\end{table}
%
% END ARISTO NATURALLI ONLY
%

\Subsubsection{aristo-analysis}{Error Analysis}
