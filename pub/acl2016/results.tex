\Section{results}{Evaluation}


%We evaluate our system in two settings:
%  we sanity check that our modifications to NaturalLI have not hindered
%  its ability to capture strict logical entailments on the FraCaS test suite,
%  and then evaluate the system on answering multiple choice science questions
%  from the Aristo dataset.
%
%%
%% FraCaS
%%
%\Subsection{fracas}{FraCaS}
%\def\a#1{\textbf{#1}}
%\begin{table}
%\begin{center}
%  \resizebox{0.48\textwidth}{!}{
%  \begin{tabular}{llcccccccc}
%    \hline
%    $\mathsection$ & Category & Count & \multicolumn{2}{c}{Precision} & \multicolumn{2}{c}{Recall} & \multicolumn{3}{c}{Accuracy} \\
%                   &          &       & \blue{N} & M08          & \blue{N} & M08          & \blue{N}  & M08 & A14 \\
%    \hline
%                          % Pme           P07   Rme          R08   Ame         A07  A08
%    \a{1} & \a{Quantifiers}  & \a{44} & \a{\blue{ }}  & \a{95}  & \a{\blue{   }} & \a{100} & \a{\blue{}} & \a{97} & \a{95} \\
%    \a{5} & \a{Adjectives}   & \a{15} & \a{\blue{ }}  & \a{71}  & \a{\blue{   }} & \a{83}  & \a{\blue{}} & \a{80} & \a{73} \\
%    \a{6} & \a{Comparatives} & \a{16} & \a{\blue{ }}  & \a{88}  & \a{\blue{   }} & \a{89}  & \a{\blue{}} & \a{81} & \a{87} \\
%    \hline                                                                                                                 
%\multicolumn{2}{l}{\a{Total}}                                                                                              
%                         & \a{75}     & \a{\blue{ }}  & \a{89}  & \a{\blue{   }}  & \a{94}  & \a{\blue{}} & \a{90} & \a{89} \\
%    \hline
%  \end{tabular}
%  }
%  \caption{
%    \label{tab:fracas}
%    Results on the FraCaS textual entailment suite.
%    N is this work; M08 is \newcite{key:2008maccartney-natlog};
%    A14 is the original NaturalLI implementation in
%      \newcite{key:2014angeli-naturalli}.
%    Note that these results assume gold parse trees, and
%      that following prior work this is not a blind test set.
%  }
%\end{center}
%\end{table}
%
%% Intro to FraCaS
%The FraCaS corpus \cite{key:1996cooper-fracas}
%  is a small corpus of entailment problems, aimed at providing a
%  comprehensive test of a system's handling of various
%  entailment patterns.
%We process the corpus following \newcite{key:2007maccartney-natlog} and
%  \newcite{key:2014angeli-naturalli}.
%Results are reported in \reftab{fracas}.
%
%% Qualify results
%The results confirm that moving to dependency trees has not hindered the
%  system's ability to capture valid entailments.
%Although, it should be noted that the evaluation assumes gold dependency
%  trees, and following prior work the test suite is not a blind test
%  set.

%
% Aristo
%
% What is Aristo?
We evaluate our entailment system on the Regents Science Exam portion of
  the Aristo dataset \cite{key:2013clark-aristo,key:2015clark-aristo}.
The dataset consists of a collection of multiple-choice science questions
  from the New York Regents 4$^{\textrm{th}}$ Grade Science Exams
  \cite{key:NYSED}.
Each multiple choice option is translated to a candidate hypotheses.
A large corpus is given as a knowledge base;
  the task is to find support in this knowledge base for the hypothesis.

% Why do we use it?
Our system is in many ways well-suited to the dataset.
Although certainly many of the facts require complex reasoning
  (see \refsec{aristo-analysis}), the majority can be
  answered from a single premise.
%  -- allowing Natural Logic to be used
%  as the inference formalism.
Unlike FraCaS or the RTE challenges, however, the task does not have explicit
  premises to run inference from, but rather must infer the truth of the
  hypothesis from a large collection of supporting text.
%This makes NaturalLI a good candidate for an approach.
%However, the queries in the Aristo dataset are relatively longer, complete
%  sentences than the common sense reasoning task in
%  \newcite{key:2014angeli-naturalli}, necessitating the 
%  improvements in \refsecs{naturalli}{softsignal}.
%Unlike the common-sense reasoning task in \newcite{key:2014angeli-naturalli}
%  though,
%  the queries in the Aristo dataset are relatively longer, complete sentences.
%This necessitates the 

\Subsection{aristo-data}{Data Processing}
% Corpora
We make use of two collections of unlabeled corpora for our experiments.
The first of these is the Barron's study guide (\textsc{Barron's}), 
  consisting of \num{1200} sentences.
This is the corpus used by \newcite{key:2015hixon-aristo} for their conversational
  dialog engine \knowbot, and therefore constitutes a more fair comparison against 
  their results.
However, we also make use of the full \textsc{Scitext} corpus \cite{key:2014clark-aristo}.
This corpus consists of \num{1316278} supporting sentences, 
  including the Barron's study guide alongside 
  simple Wikipedia, dictionaries, and a science textbook.

% Pre-processing
Since we lose all document context when searching over the corpus with NaturalLI,
  we first pre-process the corpus to resolve high-precision cases of
  pronominal coreference, via a set of very simple high-precision sieves.
This finds the most recent candidate antecedent (NP or named entity)
  which, in order of preference, matches either the pronoun's animacy, 
  gender, and number.
%The corpus is then filtered for non-ascii characters (yielding \num{945956} sentences),
%  and exact duplicate sentences are removed.
%This results in a total of \num{822748} facts in the supporting corpus.
Filtering to remove duplicate sentences and sentences containing
    non-ASCII characters yields a total of \num{822748} facts in the corpus.

% Lucene
These sentences were then indexed using Solr.
The set of promising premises for the soft alignment in \refsec{softsignal}, as well as
  the Solr score feature in the lexical classifier (\refsec{softsignal-classifier}),
  were obtained by querying Solr using the default similarity metric and scoring function.
% Questions to Candidates
On the query side, questions were converted to answers using the same methodology as
  \newcite{key:2015hixon-aristo}.
In cases where the question contained multiple sentences, only the last sentence
  was considered.
As discussed in \refsec{aristo-analysis}, 
  we do not attempt reasoning over multiple sentences, and the last sentence is
  likely the most informative sentence in a longer passage.

%
% Mutliple Choice as Entailment
%
\Subsection{aristo-train}{Training an Entailment Classifier}
To train a soft entailment classifier, we needed a set of positive
  and negative entailment instances.
These were collected on Mechanical Turk. % from a set of candidate entailment pairs.
In particular, for each true hypothesis in the training set and for each sentence
  in the Barron's study guide, we found the top 8 results from Solr and considered
  these to be candidate entailments.
These were then shown to Turkers, who decided whether the premise entailed the
  hypothesis, the hypothesis entailed the premise, both, or neither.
Note that each pair was shown to only one Turker, lowering the cost of
  data collection, but consequently resulting in a somewhat noisy dataset.
The data was augmented with additional negatives, collected by taking
  the top 10 Solr results for each false hypothesis in the training set.
This yielded a total of \num{21306} examples.
%which was used to train a simple logistic regression classifier.

The scores returned from NaturalLI incorporate negation in two ways:
  if NaturalLI finds a contradictory premise, the score is set to zero.
If NaturalLI finds a soft negation (see \refsec{softsignal-keywords}),
  and did not find an explicit supporting premise, the score is discounted
  by 0.75 -- a value tuned on the training set.
For all systems, any premise which did not contain the candidate answer to the
  multiple choice query was discounted by a value tuned on the training
  set.

%Note that despite these, the system is relatively domain agnostic -- no
%  lexical features or domain-specific reasoning is employed.


% Some definitions
\def\t#1{\footnotesize{#1}}
\def\b#1{\t{\textbf{#1}}}
\def\m#1{\t{\textcolor{darkblue}{#1}}}
\def\c#1{\b{\textcolor{darkblue}{#1}}}
\def\colspaceS{2.0mm}
\def\colspaceM{3.0mm}
\def\colspaceL{5.0mm}

%
% RESULTS TABLE
%
% The table
\def\arraystretch{0.85}
\begin{table}
\begin{center}
\begin{tabular}{l@{\hskip \colspaceL}c@{\hskip \colspaceS}c@{\hskip \colspaceL}c@{\hskip \colspaceS}c}
\toprule
\textbf{\t{System}} & \multicolumn{2}{l}{\textbf{\t{Barron's}}} & \multicolumn{2}{l}{\textbf{\textsc{\t{Scitext}}}} \\
 & \t{Train} & \t{Test} & \t{Train} & \t{Test} \\     %barrons              %all
\toprule                          %train    %test      %train    %test
\t{\textsc{Knowbot} (held-out)} & \t{45}  & \t{--}   & \t{--}  & \t{--} \\
\t{\textsc{Knowbot} (oracle)}   & \t{57}  & \t{--}   & \t{--}  & \t{--} \\
\midrule                                                           
\t{Solr Only}                   & \t{49}  & \t{42}   & \t{62}  & \t{58} \\
\t{Classifier}                  & \t{53}  & \t{52}   & \t{68}  & \t{60} \\
\t{$~~$ + Solr}                 & \t{53}  & \t{48}   & \t{66}  & \t{64} \\
\midrule                                                           
\t{Evaluation Function}         & \t{52}  & \b{54}   & \t{61}  & \t{63} \\
\t{$~~$ + Solr}                 & \t{50}  & \t{45}   & \t{62}  & \t{58} \\
\m{NaturalLI}                   & \m{52}  & \m{51}   & \m{65}  & \m{61} \\
\m{$~~$ + Solr}                 & \c{55}  & \m{49}   & \m{73}  & \m{61} \\
\m{$~~$ + Solr + Classifier}    & \c{55}  & \m{49}   & \c{74}  & \c{67} \\
\bottomrule
\end{tabular}
\end{center}
% The caption
\caption{
\label{tab:aristonaturalli}
Accuracy on the Aristo science questions dataset.
All NaturalLI runs include the evaluation function.
Results are reported using only the Barron's study guide or \textsc{Scitext}
  as the supporting
\textsc{Knowbot} is the dialog system presented in Hixon et. al (2015). % caption doesn't like \cite
The held-out version uses additional facts from other question's dialogs;
  the oracle version made use of human input on the question it was 
  answering.
The test set did not exist at the time \textsc{Knowbot} was published.
}
\end{table}
\nocite{key:2015hixon-aristo}  % for the inline reference above

%
% Results
%
\Subsection{aristo-results}{Experimental Results}
We present results on the Aristo dataset in \reftab{aristonaturalli},
  alongside prior work and strong baselines.
In all cases, NaturalLI is run with the evaluation function enabled;
  the limited size of the text corpus and the complexity of the questions
  would cause the basic NaturalLI system to perform poorly.
The test set for this corpus consists of only 68 examples,
  and therefore both perceived large differences in model scores 
  and the apparent best system should be interpreted cautiously.
NaturalLI consistently achieves the best training accuracy,
  and is more stable between configurations on the test set.
For instance,
  it may be consistently discarding lexically similar but actually contradictory
  premises that often confuse some subset of the baselines.

% Prior work
\textsc{Knowbot} is the dialog system presented in 
  \newcite{key:2015hixon-aristo}.
We report numbers for two variants of the system:
  \textit{held-out} is the system's performance when it is not allowed
  to use the dialog collected from humans for the example it is answering;
  \textit{oracle} is the full system.
Note that the \textit{oracle} variant is a human-in-the-loop system.

% Baselines
We additionally present three baselines.
The first simply uses Solr's IR confidence to rank
  entailment (\textit{Solr Only} in \reftab{aristonaturalli}).
The max IR score of any premise given a hypothesis is taken
  as the score for that hypothesis.
Furthermore, we report results for the entailment classifier defined
  in \refsec{softsignal-classifier} (\textit{Classifier}), optionally
  including the Solr score as a feature.
We also report performance of the evaluation function in NaturalLI
  applied directly to the premise and hypothesis, without any inference
  (\textit{Evaluation Function}).

% Evaluate NaturalLI
Last, we evaluate NaturalLI with the improvements presented in this paper
  (\textit{NaturalLI} in \reftab{aristonaturalli}).
We additionally tune weights on our training set for a 
  simple model combination with
  (1) Solr (with weight 6:1 for NaturalLI) and 
  (2) the standalone classifier (with weight 24:1 for NaturalLI).
%  as well as as a model which takes a weighted combination of the scores
%  produced by NaturalLI and Solr, tuned on the training set to
%  a weight of 6.0 to NaturalLI.
Empirically, both parameters were observed to be fairly robust.

To demonstrate the system's robustness on a larger dataset, we additionally
  evaluate on a test set of 250 additional science exam questions, with an
  associated 500 example training set (and 249 example development set).
These are substantially more difficult as they contain a far larger number
  of questions that require an understanding of a more complex process.
Nonetheless, the trend illustrated in \reftab{aristonaturalli} holds
  for this larger set, given in \reftab{big-dataset}.
Note that with a web-scale corpus, accuracy of an IR-based
  system can be pushed up to 51.4\%; a PMI-based solver, in turn,
  achieves an accuracy
  of 54.8\% -- admittedly higher than our best 
  system \cite{key:2016clark-aristo}.\footnote{
    Results from personal correspondence with the authors.
  }
An interesting avenue of future work would be to run NaturalLI
  over such a large web-scale corpus, and to incorporate PMI-based
  statistics into the evaluation function.

\begin{table}[t]
\begin{center}
\begin{tabular}{lc}
\toprule
\textbf{System} & \textbf{Test Accuracy} \\
\midrule
Solr Only  & 46.8 \\
Classifier & 43.6 \\
\midrule
NaturalLI & 46.4 \\
$~~$ + Solr & \textbf{48.0} \\
\bottomrule
\end{tabular}
\end{center}
\caption{\label{tab:big-dataset}
Results of our baselines and NaturalLI on a larger dataset of 250 examples.
All NaturalLI runs include the evaluation function.
}
\end{table}


%Results on the Aristo dataset are reported in \reftab{aristonaturalli}.
%\textit{NaturalLI} denotes our system, using NaturalLI inferences incorporating
%  the soft classifier.
%%Note that this already performs as well as prior work published on the dataset,
%%  and scales gracefully to a larger supporting corpus.
%%
%We additionally report results for a number of traditional baselines.
%The simplest of these is to rank answers based on the Solr score returned by
%  Solr.
%This turned out to be a surprisingly strong baseline -- furthermore, as the supporting
%  corpus size increases, the scores become significantly better calibrated, and
%  performance improves dramatically.
%
%Our second baseline uses only the keyword overlap classifier, either with or without
%  the Solr scores included as a feature.
%Both of these baselines offer a boost over using the Solr score alone -- although
%  it's interesting to note that the Solr score actively hurts the classifier's
%  performance if the corpus size is small (i.e, when run on the Barron's corpus).
%
%Lastly, we present results for the combination of NaturalLI with a soft classifier,
%  and show that it outperforms both prior work and the above baselines.
%In fact, on the training set, the classifier can nearly pass the exam with a score
%  of \todo{hopefully 70?}.



%
% DISCUSSION
%
\Subsection{aristo-analysis}{Discussion}

% Multi-premise sentences
We analyze some common types of errors made by the system on the training set.
The most common error can be attributed to the question requiring complex reasoning
  about multiple premises.
\num{29} of \num{108} questions in the training set (26\%) contain multiple
  premises.
Some of these cases can be recovered from 
  (e.g., \w{This happens because the smooth road has less friction.}), 
  while others are trivially out of scope for our method 
  (e.g., \w{The volume of water most likely decreased.}).
Although there is usually still some signal for which answer is most likely to be correct,
  these questions are fundamentally out-of-scope for the approach.
%Empirically, we correctly predict \todo{XYZ} of these questions, performing somewhat 
%  but not substantially above random chance.

Another class of errors which deserves mention are cases where a system produces
  the same score for multiple answers.
This occurs fairly frequently in the standalone classifier 
  (7\% of examples in training; 4\% loss from random guesses),
  and especially often in NaturalLI (11\%; 6\% loss from random guesses).
This offers some insight into why incorporating other models -- even with
  low weight -- can offer significant boosts in the performance
  of NaturalLI.
Both this and the previous class could be further mitigated
  by having a notion of a \textit{process}; e.g., as in
  \newcite{key:2014berant-bioprocess}.
%Empirically, NaturalLI is forced to guess on 33\% more examples than the
%  alignment feature classifier.
%This alone accounts for 6.5 points of accuracy -- on errors which are often


% Other errors
Other questions are simply not supported by any single sentence in the corpus.
For example, \w{A human offspring can inherit blue eyes} has
  no support in the corpus that does not require significant multi-step inferences.

A remaining chunk of errors are simply classification errors.
For example, \w{Water freezing is an example of a gas changing to a solid}
  is marked as the best hypothesis, supported incorrectly by 
  \w{An ice cube is an example of matter that changes from a solid to a liquid to a gas},
  which after mutating \w{water} to \w{ice cube} matches every keyword in the
  hypothesis.
  

%% Drop from align features to NaturalLI?
%A peculiarity of the results is the drop in accuracy from adding the alignment
%  features to NaturalLI.
%We hypothesize that this is a result of missing a few alignments that
%  are not trivial from tokens alone -- e.g., aligning \w{freezing water} and
%  \w{water}.
%Empirically, NaturalLI is forced to guess on 33\% more examples than the
%  alignment feature classifier.
%This alone accounts for 6.5 points of accuracy -- on errors which are often
%  easily corrected by any additional signal (in this case, Solr).
